\documentclass{beamer}
\usepackage{graphicx}
\usepackage{times}
\usepackage[polish]{babel}
\usepackage{polski}
\usepackage[utf8]{inputenc}

% dodatkowe pakiety
\usepackage{enumerate}
% Listings ------------------------------------------------------------
\usepackage{listings}
% Scala listings
% "define" Scala
\lstdefinelanguage{scala} {
  morekeywords={abstract,case,catch,class,def,%
    do,else,extends,false,final,finally,%
    for,if,implicit,import,match,mixin,%
    new,null,object,override,package,%
    private,protected,requires,return,sealed,%
    super,this,throw,trait,true,try,%
    type,val,var,while,with,yield},
  otherkeywords={=>,<-,<\%,<:,>:,\#,@},
  sensitive=true,
  morecomment=[l]{//},
  morecomment=[n]{/*}{*/},
  morestring=[b]",
  morestring=[b]',
  morestring=[b]"""
}

% IntelliJ Colors for listings
\usepackage{color}
\definecolor{dkgreen}{rgb}{0,0.6,0}
\definecolor{gray}{rgb}{0.5,0.5,0.5}
\definecolor{mauve}{rgb}{0.58,0,0.82}
 
% Default settings for code listings
\lstset{
  frame=tb,
  language=Scala,
  aboveskip=3mm,
  belowskip=3mm,
  showstringspaces=false,
  columns=flexible,
  basicstyle={\small\ttfamily},
  numbers=none,
  numberstyle=\tiny\color{gray},
  keywordstyle=\color{blue},
  commentstyle=\color{gray},
  stringstyle=\color{dkgreen},
  frame=single,
  breaklines=true,
  breakatwhitespace=true
  tabsize=3
}

\lstloadlanguages{TeX}

\usetheme{AGH}

\title[ProtoDoc - odpowiednik JavaDoc dla Google Protocol Buffers]{ProtoDoc - \\ odpowiednik JavaDoc\\ dla Google Protocol Buffers}

\author[K. Malawski]{Konrad Malawski}

\date[2011]{22.01.2011}

\institute[AGH-UST]
{Faculty of EEACSE\\ 
Department of Automatics
}

\setbeamertemplate{itemize item}{$\maltese$}

\begin{document}

%---------------------------------------------------------------------------


{
\usebackgroundtemplate{\includegraphics[width=\paperwidth]{titlepage}} % wersja angielska
%\usebackgroundtemplate{\includegraphics[width=\paperwidth]{titlepagepl}} % wersja polska
 \begin{frame}
   \titlepage
 \end{frame}
}

%---------------------------------------------------------------------------


\begin{frame}
\frametitle{Plan prezentacji}
 \begin{block}{Plan prezentacji}
  \begin{itemize}
   \item omówienie problemu
   \item omówienie zastosowanego rozwiązania
   \item porównanie z innymi parserami
   \item prezentacja wyniku działania aplikacji
   \item podsumowanie
  \end{itemize}
 \end{block}
\end{frame}

%---------------------------------------------------------------------------


\begin{frame}
\frametitle{Some AGH-UST Faculties}

\begin{itemize}
\item Faculty of Mining and Geoengineering
\item Faculty of Metals Engineering and Industrial Computer Science
\item Faculty of Electrical Engineering, Automatics, Computer Science and Electronics
\item Faculty of Mechanical Engineering and Robotics
\item Faculty of Geology, Geophysics and Environment Protection
\item \dots
\end{itemize}

\end{frame}

%---------------------------------------------------------------------------


\begin{frame}
\frametitle{Scala Parser Combinators}

\begin{columns}
\column{0.45\textwidth}
 \begin{block}{Za}
  
 \end{block}

\column{0.45\textwidth}
 \begin{block}{Przeciw}
  
 \end{block}
\end{columns}

\end{frame}


%---------------------------------------------------------------------------


\begin{frame}[fragile]
\frametitle{Fragment kodu parsera}

\begin{verbatim}
  def messageTypeDef: Parser[ProtoMessageType] = opt(comment) ~ "message" ~ ID ~ "{" ~ rep(enumTypeDef | instanceField | messageTypeDef) ~ "}" ^^ {
    case maybeDoc ~ m ~ id ~ p1 ~ allFields ~ p2 =>
      val comment = maybeDoc.getOrElse("")
      val pack = ""

      val processedFields = addOuterMessageInfo(id, pack, allFields) 

      new ProtoMessageType(messageName = id,
                           packageName = pack,
                           fields = processedFields /*will be implicitly filtered*/ ,
                           enums = processedFields /*will be implicitly filtered*/ ,
                           innerMessages = processedFields /*will be implicitly filtered*/)
  }
\end{verbatim}


\end{frame}

%---------------------------------------------------------------------------


\begin{frame}
\frametitle{Podsumowanie}


\end{frame}

%---------------------------------------------------------------------------


\end{document}

