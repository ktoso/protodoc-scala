\chapter{Podstawy języka Scala}
\label{cha:appendixA}
Celem tego dodatku jest przybliżenie czytelnikowi języka ,,Scala'' aby w wystarczająco płynny sposób mógł czytać przykłady kodu używane w tym dokumencie.

%---------------------------------------------------------------------------
\section{Krótka historia języka}
\label{sec:scala_history}
Język Scala (,,Scalable Language'') najłatwiej jest przedstawić jako hybrydę dwóch znanych nurtów programowania: programowania obiektowego oraz funkcyjnego, wraz z 
powiązanymi z nimi językami programowania. Twórca języka Scala, Martin Oderski\cite{odersky_scala}

Jako konkretnych ,,rodziców'' można by wskazać: 
\begin{itemize}
 \item \textbf{Java} - jako reprezentant nurtu obiektowego 
 \item oraz języki: \textbf{Haskell}, \textbf{SML} oraz pewne elementy języka \textbf{Erlang} (głównie \textit{Actor model}).
\end{itemize}


%---------------------------------------------------------------------------
\section{Podstawy}
\label{sec:scala_basics}
Ta sekcja służy przybliżeniu czyletnikowi języka \textit{Scala} na poziomie wystatczającym aby swobodnie czytać przykłady
kodu umieszczone w tej pracy. W niektórych przykładach pomijane są przypadki skrajne lub nietypowe, celem szybkiego oraz 
jasnego przedstawienia minimum wiedzy na temat języka aby móc swobodnie go ,,czytać''.

\textit{Scala} jest językiem statycznie typowanym posiadającym lokalne ,,Type Inferrence''. Pozwala to kompilatorowi 
\textit{scalac} na ,,odnajdywanie'' typów wszystkich zmiennych oraz typów zwracanych przez metody podczas kompilacji,
bez potrzeby definiowania ich wprost. System ten

Użycie nawiasów \verb|()|, średnika \verb|;| oraz kropki \verb|.| jest analogiczne jak w przypadku Javy, 
jednak w wielu przypadkach opcjonalne gdyż kompilator jest w stanie wydedukować gdzie powinny się znaleźć.
\begin{lstlisting}
 val value = Option(42);
 val other = value.orElse(0);

 // moze zostac zastąpione
 val value = Option(42)
 val other = value orElse 0
\end{lstlisting}

Jednym z ciekawych przykładów stosowania notacji bez nawiasów i kropek jest \textit{ScalaTest}\footnote{ScalaTest - framework do testowania - http://www.scalatest.org}
(przy którego pomocy pisano testy w tym projekcie). Przykładowa \textit{asercja} napisana w \textit{DSL}u definiowanym przez tą bibliotekę wygląda następująco:
\begin{lstlisting}
 messages should (contain key ("Has") and not contain value ("NoSuchMsg"))
\end{lstlisting}



%---------------------------------------------------------------------------
\section{Scala Parser Combinators}
\label{sec:scala_parser_combinators}
