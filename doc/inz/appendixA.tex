\chapter{Podstawy języka Scala}
\label{cha:appendixA}
Celem tego dodatku jest przybliżenie czytelnikowi języka ,,Scala'' aby w wystarczająco płynny sposób mógł czytać przykłady kodu używane w tym dokumencie.

%---------------------------------------------------------------------------
\section{Krótka historia języka}
\label{sec:scala_history}
Język Scala (,,Scalable Language'') najłatwiej jest przedstawić jako hybrydę dwóch znanych nurtów programowania: programowania obiektowego oraz funkcyjnego, wraz z 
powiązanymi z nimi językami programowania. Twórca języka Scala, Martin Oderski\cite{odersky_scala}

Jako konkretnych ,,rodziców'' można by wskazać: 
\begin{itemize}
 \item \textbf{Java} - jako reprezentant nurtu obiektowego 
 \item oraz języki: \textbf{Haskell}, \textbf{SML} oraz pewne elementy języka \textbf{Erlang} (głównie \textit{Actor model}).
\end{itemize}


%---------------------------------------------------------------------------
\section{Podstawy}
\label{sec:scala_basics}

%---------------------------------------------------------------------------
\section{Scala Parser Combinators}
\label{sec:scala_parser_combinators}
