\documentclass[pdflatex,11pt]{aghdpl}
% \documentclass{aghdpl}               % przy kompilacji programem latex
% \documentclass[pdflatex,en]{aghdpl}  % praca w jêzyku angielskim
\usepackage[polish]{babel}
\usepackage{polski}
\usepackage[utf8]{inputenc}

% dodatkowe pakiety
\usepackage{enumerate}
\usepackage{hyperref}
% Listings ------------------------------------------------------------
\usepackage{listings}
% Scala listings
% "define" Scala
\lstdefinelanguage{scala}{
  morekeywords={abstract,case,catch,class,def,%
    do,else,extends,false,final,finally,%
    for,if,implicit,import,match,mixin,%
    new,null,object,override,package,%
    private,protected,requires,return,sealed,%
    super,this,throw,trait,true,try,%
    type,val,var,while,with,yield},
  otherkeywords={=>,<-,<\%,<:,>:,\#,@},
  sensitive=true,
  morecomment=[l]{//},
  morecomment=[n]{/*}{*/},
  morestring=[b]",
  morestring=[b]',
  morestring=[b]"""
}

% IntelliJ Colors for listings
\usepackage{color}
\definecolor{dkgreen}{rgb}{0,0.6,0}
\definecolor{gray}{rgb}{0.5,0.5,0.5}
\definecolor{mauve}{rgb}{0.58,0,0.82}
 
% Default settings for code listings
\lstset{
  frame=tb,
  language=Scala,
  aboveskip=3mm,
  belowskip=3mm,
  showstringspaces=false,
  columns=flexible,
  basicstyle={\small\ttfamily},
  numbers=none,
  numberstyle=\tiny\color{gray},
  keywordstyle=\color{blue},
  commentstyle=\color{gray},
  stringstyle=\color{dkgreen},
  frame=single,
  breaklines=true,
  breakatwhitespace=true
  tabsize=3
}

\lstloadlanguages{TeX}

%---------------------------------------------------------------------------

\author{Konrad Malawski}
\shortauthor{K. Malawski}

\titlePL{ProtoDoc\\Implementacja odpowiednika narzędzia JavaDoc dla jęzka definicji interfejsów Google~Protocol~Buffers}
\titleEN{ProtoDoc\\Development of a JavaDoc tool equivalent for the Google Protocol Buffers Interface~Description~Language}

\shorttitlePL{ProtoDoc - impl. odpowiednika JavaDoc dla Google~Protocol~Buffers} 
\shorttitleEN{ProtoDoc - impl. of JavaDoc like tool for Google~Protocol~Buffers}

\thesistypePL{Praca inżynierska}
\thesistypeEN{Bachelor of Science Thesis}

\supervisorPL{dr inż. Jacek Piwowarczyk}
\supervisorEN{Jacek Piwowarczyk Ph.D}

\date{2011}

\departmentPL{Katedra Automatyki}
\departmentEN{Department of Automatics}

\facultyPL{Wydział Elektrotechniki, Automatyki, Informatyki i Elektroniki}
\facultyEN{Faculty of Electrical Engineering, Automatics, Computer Science and Electronics}

\acknowledgements{} % TODO podziękowania

\setlength{\cftsecnumwidth}{10mm}

%---------------------------------------------------------------------------

\begin{document}

\titlepages

\tableofcontents
\clearpage

%---------------------------------------------------------------------------------------------------------------------------------------------------------------------------
%\chapter{Wprowadzenie}
\label{cha:wprowadzenie}

%---------------------------------------------------------------------------
\section{Cele pracy}
\label{sec:celePracy}


%---------------------------------------------------------------------------
\section{Zawartość pracy}
\label{sec:zawartoscPracy}

%---------------------------------------------------------------------------
\section{Zrzuty ekranu wygenerowanej dokumentacji}
\label{sec:screenshots}

\begin{center}
 \includegraphics[width=\textwidth]{../protodoc_main.png}
 % protodoc_main.png: 949x970 pixel, 93dpi, 25.92x26.50 cm, bb=0 0 735 751
\end{center}


\chapter{Wprowadzenie}
\label{cha:wprowadzenie}

%---------------------------------------------------------------------------
\section{Cel pracy}
\label{sec:celePracy}

Celem projektu jest implementacja narzędzia generującego dokumentację na podstawie plików 
*.proto zawierających zapisane przy pomocy ,,języka definicji interfejsów'' (tzw. \textit{Interface Description Language}, w skrócie \textit{IDL}) - Google Protocol Buffers. 


Potrzebę implementacji takiego narzędzia motywuję doświadczeniem w pracy z Protocol Buffers, gdy mamy do czynienia z dużą ilością plików *.proto (setki). 
Brak automatycznie generowanej dokumentacji tak dużego zbioru wiadomości znacznie utrudniał zapoznanie się z systemem oraz przystąpienie do sprawnej pracy z nim.
Gdyby taka, zawsze aktualna, dokumentacja była dostępna w firmowym intranecie na przykład, komunikacja między zespołami o wiadomościach byłaby znacznie prostsza - 
możliwe byłoby wówczas przesłanie sobie między programistami linku do właściwej wiadomości ,,to tej wiadomości szukasz'', włącznie z upewnieniem się, że na pewno
wskazana wiadomość nie jest przestarzała - zawierałaby wówczas odnośnik do wiadomości którą obecnie powinno się stosować.


Proces generowania dokumentacji jest analogiczny do znanego z świata Javy narzędzia JavaDoc 
\footnote[1]{JavaDoc - Strona domowa projektu: \href{http://bit.ly/javadochome}{http://bit.ly/javadochome}} - stąd zainspirowana JavaDociem nazwa tego projektu. 
Sam proces generowania dokumentacji polega na dostarczeniu narzędziu plikół *.proto, które następnie są parsowane oraz na podstawie tego procesu, 
generowana jest strona www zawierające wszystkie zebrane informacje, włącznie z komentarzami oraz dodatkowymi informacjami 
typu ,,\textit{deprecated}'' (ang. przestarzałe). Jako dodatkowy krok wygenerowana strona mogłaby automatycznie zostać opublikowana w firmowym intranecie.

Cały proces możliwe jest w pełni zintegrować z narzędziami stosowanymi do budowania projektów np. Javowych. W przypadku projektów Javowych, obecnym \textit{de facto}
standardem w wielu firmach stał się Apache Maven \footnote[2]{Apache Maven - Strona domowa projektu: \href{http://maven.apache.org}{http://maven.apache.org}}.
ProtoDoc może zostać użyty razem z Maven aby automatycznie, podczas budowania projektu generować
dokumentację. Możliwe jest uruchomienie tego zadania samodzielnie, lub jako jeden z etapów budowy projektu - dzięki czemu nie konieczne jest pamiętanie oraz ręczne
aktualizowanie dokumentacji - byłaby automatycznie generowana podczas buildu, na przykład na serwerze ciągłej integracji.

\newpage

%---------------------------------------------------------------------------
\section{Analiza obecnie dostępnych rozwiązań}
\label{sec:dostepneNarzedzia}


\begin{center}
  \begin{tabular}{ | p{\textwidth} |}
    \hline
      Celem ułatwienia zrozumienia poniższego, oraz kolejnych rozdziałów w przypadku gdy czytelnik nie miał 
      jeszcze styczności z Google Protocol Buffers zalecane jest wpierw
      zapoznanie się z \textit{Dodatkiem \ref{cha:appendixA}}, gdzie wyjaśniane jest dokładnie jak oraz dlaczego działa ProtoBuf\footnote[1]{ProtoBuf - oficjalna skrótowa nazwa na ,,Protocol Buffers''}. \\ \hline
  \end{tabular}
\end{center}


Niestety na chwilę obecną nie są dostępne narzędzia pozwalające na generowanie dokumentacji z plików Protocol Buffers.
\textit{Analiza obecnych rozwiązań zatem organiczy się do rozważenia opłacalności wykorzystania jakiegoś projektu open source jako bazy dla ProtoDoc.}

~\\\*

Jak się okaże, najopłóacalniejsza z perspektywy programisty jak i użytkownika końcowego gotowej aplikacji ProtoDoc, będzie implementacja parsera,
przy wykorzystaniu języka Scala, a nie wykorzystanie istniejących rozwiązań - które na przykład posiadają bardzo duże zewnętrzne zależności, lub
ich dopasowanie do potrzeb tego projektu byłby zbyt dużym przedsięwzięciem.

\subsection{Google Protoc}

Protoc jest ,,oryginalnym'' kompilatorem plików *.proto. Zawiera ręcznie zaimplenentowany przez inżynierów google skaner oraz parser,
potrafiący obsłużyć 100\% specyfikacji ProtoBuf. Jego źródła są dostępne na stronie Google Code: \href{http://code.google.com/p/protobuf/source/browse/}{http://code.google.com/p/protobuf/source/browse/}
Projekt objęty jest licencją \textit{New BSD License\footnote{New BSD License, znana również jako 2-clause BSD license - \href{http://www.opensource.org/licenses/bsd-license.php}{http://www.opensource.org/licenses/bsd-license.php}}}.

Warto również uwypuklić pewien problem z udostępnianym przez Google kompilatorem Protocol Buffers IDL - \textit{protoc}.
Otóź nawet jeżeli źródłowy plik *.proto posiada komentarze, kompilator \textit{protoc} nie przeniesie je do wynikowych plików, np. *.java.
Parser ten niestety ignoruje całkowicie komentarze. 

Po wstępnej analizie kodu parsera dostarczanego przez Google doszedłem do wniosku, 
że niestety wykorzstanie go jako bazy ProtoDoc nie byłoby opłacalne, ze względu na bardzo dużą ilość zmian które trzeba by wprowadzić w \textit{core} parsera
 - zaimplementowanego ,,ręcznie'', bez zastosowania znanych generatorów parserów, w C++.

\subsection{Idea plugin protobuf}

Innym projektem open source zzawierającym zaimplementowany parser ProtoBuf jest plugin do ,,IntelliJ IDEA'', popularnego w świecie programistów JVM 
IDE programistycznego. Źródła znajdują się na Google Code pod adresem: \href{http://code.google.com/p/idea-plugin-protobuf/source/browse}{http://code.google.com/p/idea-plugin-protobuf/source/browse}
Projekt udostępniany jest na warunkach \textit{Apache 2.0 License}\footnote{Apache 2.0 License - \href{http://www.apache.org/licenses/LICENSE-2.0}{http://www.apache.org/licenses/LICENSE-2.0}}.

Z perspektywy ProtoDoc, interesującymi fragmentami tego projektu jest skaner oraz parser. 
Skaner jest generowany przy pomocy \textit{JFlex} \footnote{JFlex (Fast Lexical Analyzer for Java)- Strona domowa projektu: \href{http://jflex.de/}{http://jflex.de/}},
, odpowiednika narzędzia GNU Flex, dla języka Java. Skaner teoretycznie nadawałby się do ponownego wykorzystania - obsługiwane są tutaj również komentarze.

\newpage

Niestety druga z interesujących nas części aplikacji, parser, jest \textit{ściśle związany ze środowiskiem IntelliJ IDEA}, dla którego to powstał ten projekt.
IntelliJ dostarcza własny mechanizm parsowania do którego pluginy jedynie mogą się podpinać, oraz pomagać w przeprowadzeniu parsingu pliku, nie można w tym przypadku
powiedzieć że projekt zawiera całą implementację parsera. Część źródeł IntelliJ IDEA co prawda jest otwarta, jednak skorzystanie z podejścia dołączenia całego IDE,
aby być w stanie parsować pliki, wydaje się bardzo nie optymalna - rozmiar dystrybucji ProtoDoc stałby się bardzo duży (rzędku setek MB, z racji dołączonych 
zależności w postaci IntelliJ).

~\\\*

Jak widać, również i ten projekt nie dostarcza w pełni funkcjonalnej oraz łatwej to rozbudowania o potrzebne w projekcie ProtoDoc funkcjonalności implementacji parsera
Protocol Buffers. W związku z powyższym, postanowiłem wybrać sposób własnoręcznej implementacji parsera, aby proces był jednak możliwie przyjemny, oraz
możliwy do utrzymania w przyszłości - na przykład przez społeczność Open Source. Ostatecznie wybrana przezemnie technika implementacij parsera 
zostanie przedstawiona w kolejnej sekcji.



\subsection{Wybór własnoręcznej implementacji Parsera - Parser Combinators}
\label{sec:wybor_parsera}

Podsumowując, istnieją implementacje parserów Protocol Buffers na wolnych (jak wolność) licencjach, jednak rozbudowa ich o pożądane funkcjonalności,
albo byłaby zbyt czasochłonna by nazwać ją opłacalnym (zmiany manualnie implementowanego parsera \textit{protoc}) lub wymagałyby 
przepisania parsera w całości, w powodu korzystania przez nie z zewnętrznych zależności których nie da się w prosty sposób dostarczyć.
%TODO ref nie dziala!!!
Tabela \ref{tab:parers} przedstawia małe podsumowanie zastosowanych technik implementacji parserów w omawianych projektach.

\begin{table}[ch]
  \begin{center}
    \begin{tabular}{| l | l | l |}
      \hline
      Projekt & Metoda impl. skanera & Metoda impl. parsera\\
      \hline
      Google Protoc & "manualnie", C++ & "manualnie", C++\\
      \hline
      Idea-Plugin-Proto & JFlex, Java & dostarczany z IntelliJ, Java\\
      \hline
    \end{tabular}
    \caption{Zestawienie sposobów implementacji parserów w rozważanych projektach open source}
  \end{center}
  \label{tab:parers}
\end{table}

Po przeanalizowaniu powyższych projektów i porzuceniu pomysłu rozwinięci istniejącej już implementacji o potrzebne elementy, rozpocząłem
wybór generatora parserów / skanerów który chciałbym zastosować podczas tego projektu. 

Pierwotnym kandydatem do zastosowania jako generator parsera był powszechnie znany \textit{GNU~Bison} \footnote{GNU Bison - Strona domowa projektu: \href{http://www.gnu.org/software/bison}{http://www.gnu.org/software/bison}},
który w połączeniu z Flexem pozwolił na wygenerowanie parsera w ,,znajomy'' sposób. Oba te narzędzia są dobrze znane oraz sprawdzone od wielu lat oraz posiadają dobrą dokumentację.
W ramach poszukiwań innych rozwiązań natknąłem się jednak na tak zwane ,,kombinatory parserów'', a następnie na fakt iż istnieje ich implementacja wewnątrz
biblioteki standardowej języka Scala.

~\\\*

\textit{Scala} jest statycznie typowanym językiem programowania na platformę Java który wspiera zarówno \textit{obiektowy} jak i \textit{funkcyjny} paradygmat programowania.
Tak zwane ,,\textit{Parser Combinators}'' o których tutaj mowa nie są ideą nową. Pojawiły się wraz z językami funkcyjnymi, a pierwsze publikacje naukowe
na ich temat można było już napotkać w 1996 roku \cite{monadparsing} w publikacji Hutton oraz Meijer.
Pojęcie ,,parsowania przy pomocy kombinatorów parserów'' najłatwiej jest wytłumaczyć jako:


\begin{quotation}
 ,,Budowanie parserów rekursywnie zstępujących poprzez modelowanie parserów jako funkcji
   i definiowanie funkcji wyższego rzędu (zwanych kombinatorami) które implementują 
   konstrukcje takie jak sekwencjonowanie, wybór oraz powtórzenie. [...]''
   \cite{monadparsing}
\end{quotation}

Będziemy mieli zatem w efekcie do czynienia a parserem ,,rekursywnie zstępującym'' (\textit{LL(k)}). Przedstawicielem generatorów tworzących tego typu parsery jest na przykład
ANTLR \footnote{ANTLR - Strona domowa projektu: \href{http://www.antlr.org/}{http://www.antlr.org/}}, opublikowany po raz pierwszy w roku 1992 jako następca \textit{Purdue Compiler Construction Tool Set} który powstał jeszcze w roku 1989 (sic).
Ten typ parserów dodaje do znanej klasy LL(k) funkcjonalność ,,wycofania się'', z dowolnej głębokości look-ahead (parser może pracować z dowolnie dużym \textit{k},
i zawsze będzie w stanie wykonać nawrot oraz wypróbować inną ścieżkę). Korzystanie z nawrotów przez parser oczywiście niesie z sobą zmniejszenie jego wydajności,
jednakw wielu przypadkach (jak choćby Protocol Buffers, które mają stosunkowo prostą gramatykę), obawa przed spadniem wydajności nie odzwierciedla się zbytnio w 
rzeczywistości. Dobrą wiadomością jest natomiast, że w \textit{Scala Parser Combinators} możemy korzystać z wersji metod z dodanym wykrzyknikiem oznaczającym,
że pragniemy aby dany fragment był faktycznie klasy \textit{LL(1)} - jeżeli gramatyka nie jest na tyle jednoznacza aby dało się uzyskać \textit{LL(1)} w danym parserze
zostaniemy powiadomieni o tym pod postacią błędu.



% RECURSIVE DESCENT BACKTRACKING PARSER
% This parser adds backtracking infrastructure to an LL(k) RD parser, which gives it the
% power to process arbitrary-sized look-ahead sets. With backtracking infrastructure in
% place, the parser can parse ahead as far as it needs to. If it fails to find a match, it can
% rewind its input and try alternate rules. This capability makes it quite a bit more pow-
% erful than the LL(k).
% OK, so it can backtrack and choose alternate rules, but is there any order to this
% process? You can specify hints for ordering in the form of syntactic predicates, as in
% ANTLR (see [2] in section 7.6). You can declaratively specify the ordering so that the
% parser can select the most appropriate rule to apply on the input stream.
% With this kind of parser, you get much more expressive grammars, called parsing
% expression grammars (PEGs). PEG is a more expressive form of grammar that extends
% ANTLR’s backtracking and syntactic predicates (see [4] in section 7.6). PEG adds oper-
% ators like & and ! that you specify within the grammar rules to implement finer con-
% trols over backtracking and parser behaviors. They also make the grammar itself way
% more expressive. You can develop parsers for PEGs in linear time using techniques like
% memorizing.




Jednym z wyróżniajączych \textit{Scala Parser Combinators} czynników jest fakt iż zamiast pisać pliki w których deklarujemy naszą gramatykę a 
sam kod źródłowy parsera jest dopiero generowany na jego podstawie w przypadku Scali i wspomnianej biblioteki zdefiniować gramatykę parsera, 
w połączeniu z blokami kodu które miałyby dokonać odpowiednich transformacji sparsowanych tokenów dokładnie w tym samym pliku który jest ,,plikiem źródłowym parsera''.
Dzięki temu oszczędzamy na ,,kroku'' generowania kodu źródłowego parsera, który dopiero później zostałby skompilowany oraz wykonany. 
Daje to ogromną przewagę podczas poszukiwania błędów w parserze - ponieważ ewentualne problemy bezpośrednio odwołują się do tego co my napisaliśmy,
a nie do odrębnego pliku który powstał na podstawie naszego pliku.

Pomimo tak wielu zalet sama struktura definicji parsera pozostaje podobna do znanej z Bisona oraz nadal jest złudnie 
podobna do notacji \textit{BNF}\footnote{Notacja BNF - ,,Backus Naur Form'', metoda zapisu reguł gramatyki kontekstowej} 
- która jest bardzo przejrzysta oraz zazwyczaj znana, lub łatwa to utworzenia podczas pisania parsera znanego języka.

Kolejną zaletą jest umieszczenie definicji tokenów w tym samym pliku co definicja skanera - ponownie unikamy kroku generowania 
skanera (na przykład przy użyciu \textit{JFlex}a). Zmniejszenie ilości miejsc gdzie konieczne jest wprowadzenie modyfikacji, jest zatem kolejną z zalet
tego podejścia.

~\\\*

\begin{table}[ch]
  \begin{center}
    \begin{tabular}{| l | l |}
      \hline
      Metoda impl. skanera & Metoda impl. parsera\\
      \hline
      \multicolumn{2}{|c|}{Parser Combinators, Scala} \\
      \hline
    \end{tabular}
    \caption{Przedstawienie jednolitości rozwiązania z zastosowaniem \textit{Scala Parser Combinators}}
  \end{center}
  \label{tab:scala_parsers_table}
\end{table}

Gdyby umieścić Parser Combinators (tak jak przedstawiono w Tabeli \ref{tab:scala_parsers_table}) na przedstawionej powyżej tabeli, z zestawieniem jak implementowana jest która część parsera,
okazałoby się że jesteśmy wyjątkowo spójni - wszystkie części implementowane są w jednym miejscu / języku / narzędziem.

Reasumując poniższe zalety są przyczyną wyboru tego podejścia do generowania parsera ponad klasyczne narzędzia typu Flex/Bison:
\begin{itemize}
 \item Brak konieczności dodatkowego kroku generowania kodu źródłowego
  \subitem dla skanera (brak osobnego pliku ze spisem tokenów)
  \subitem dla parsera (brak osobnego języka, dedykowanego definiowaniu
 \item Wykonywany kod jest bezpośrednio związany z pisaną przez nas definicją parsera, co pozwala na łatwe poszukiwanie błędów
 \item Minimalizacja miejsc w których konieczne jest wprowadzanie zmian, cała implementacja znajduje się w 1 miejscu
\end{itemize}


\begin{verbatim}
 
\end{verbatim}

Opis działania \textit{Parser Combinators} znajduje się w kolejnej sekcji, oraz w \textit{Dodatku \ref{cha:appendixB}}, gdzie wytłumaczone zostały wszystkie zasady
konstruowania parserów przy pomocy tej biblioteki.


\begin{center}
\begin{tabular}{ | p{\textwidth} | }
  \hline 
  Jeżeli czytelnik jeszcze nie miał styczności z Scalą oraz dostarczaną przez wraz z nią biblioteką 
  \textit{Parser Combinators} zalecane jest zapoznanie się z \textit{Dodatkiem \ref{cha:appendixB}}, 
  gdzie szczegółowo omówiono zasady działania samego języka jak i Parser Combinators. \\
  \hline 
\end{tabular}
\end{center}


%===========================================================================
% \chapter{Streszczone omówienie zasad działania Paser Combinators}
% \label{sec:wprowadzenieTeoretyczne}
% 
% W tym rozdziale zostaną przybliżone podstawy działania: 
% \begin{itemize}
%  \item \textbf{Scala Parser Combinators} - zastosowanego generatora parserów w tym projekcie,
%  \item oraz \textbf{Google Protocol Buffers} - parsowanego języka, oraz powiązanych z nim narzędzi. 
% \end{itemize}
% 
% Omówione zostaną podjęte decyzje, dlaczego wybrano tą a nie inną technologię, oraz jakie konsekwencje poniesiono w związku z tymi wyborami.
% 
% TODO TODO TODO TODO TODO TODO TODO TODO TODO TODO 
% 
% LUB USUNĄĆ I ZASTĄPIĆ DODATKAMI

%===========================================================================
\chapter{Szczegóły implementacyjne}
\label{sec:zastosowanePodejscie}

Przed rozpoczęciem lektury poniższego rozdziału zalecane jest zapoznanie się z podstawami języków Scala jak i narzędzia/języka Protocol Buffers. 
W sekcji tematów zostały przedstawione w ramach dodatków 
\begin{itemize}
 \item \textit{Dodatek \ref{cha:appendixA}} --- omawiający Protocol Buffers. Gdzie i dlaczego powinno się je stosować, oraz przytoczenie benchmarku potwierdzającego te tezy.
                                                Zostanie też przedstawiona składnia ProtoBuf IDL oraz workflow stosowany podczas pracy z ProtoBuf.
 \item \textit{Dodatek \ref{cha:appendixB}} --- omawiający podstawy języka \textit{Scala}, po których przeczytaniu czytelnik będzie w stanie czytać ze zrozumieniem kod
                                                projektu będącego przedmiotem tego projektu. Przedstawione zostaną również szczegóły Parser Combinators, oraz jak je czytać / pisać w języku Scala.
\end{itemize}

W tym rozdziale omówione zostaną zarówno ogólne założenia przyjęte podczas projektowania systemu, jak i struktura klas przyjęta celem
modelowania struktury typów Protocol Buffers. Następnie przedstawione zostaną poszczególne komponenty aplikacji, z naciskiem na uzasadnienie
wybranych rozwiązań oraz rozważenia ich zalet, wad oraz potencjalnych możliwości usunięcia zauważonych wad. 

Przedstawione zostaną również elementy kodów źródłowych, skrócone do postaci wystarczającej na cel omówienia danego tematu 
- w przypadku chęci zapoznania się ze całością implementacji np. komponentu parsera, zachęcam do zapoznania się 
z załączonymi do pracy plikami źródłowymi projektu.

W ostatniej sekcji (\ref{sec:prepare_env}) zostanie przedstawione jak należy przygotować środowisko pracy celem kompilowania oraz budowania projektu
ze źródeł.

%---------------------------------------------------------------------------
\newpage
\section{Projekt systemu}
\label{sec:projekt_systemu}

Przed omówieniem poszczególnych elementów implementacji \textit{ProtoDoc}, 
spójrzmy na architekturę tego narzędzia w holistyczny sposób. Główne odpowiedzialności aplikacji zostały podzielone pomiędzy
poniższe klasy (konkretniej, klasy typu \textbf{object} - omówionych dokładniej w Dodatku \ref{cha:appendixB}, dotyczącym języka Scala):

\begin{itemize}
 \item \verb|ProtoDocCompiler| - jest fasadą nad parserem, wykorzystuje dodatkowo \verb|ProtoBufVerifier| celem sprawdzania poprawności oraz rozwiązania 
                                 niejasności mogących jeszcze istnieć po pierwszym przebiegu parsera dotyczących typowania pól
 \item \verb|ProtoBufParser| - jest implementacją parsera przy pomocy Scala Parser Combinators, omówiony zostanie szczegółowo w kolejnych rozdziałach
 \item \verb|ProtoBufVerifier| - ,,weryfikator'' zajmujący się sprawdzaniem poprawności semantycznej sparsowanych plików *.proto.
                                 W przypadku napotkania błędów krytycznych, działanie protodoc może zostać w tym miejscu przerwane,
                                 oraz zwrócony zostałby komunikat informujący o przyczynie błędu.
 \item \verb|ProtoDoc|
\end{itemize}


\verb|ProtoBufParser|

\begin{figure}[ch]
\begin{center}
 \includegraphics[width=\textwidth]{main_classes.png}
\end{center}
\label{simple_visualization}
\caption{Wizualizacja interakcji pomiędzy komponentami}
\end{figure}

\begin{figure}[ch]
\begin{center}
 \includegraphics[width=\textwidth]{compile_sequence}
\end{center}
\label{sequence_diagram_parsing}
\caption{Diagram sekwencji parsowania oraz weryfikowania wiadomości}
\end{figure}




%---------------------------------------------------------------------------
\section{Parser: ProtoBufParser}
\subsection{Wprowadzenie do kombinatorów parserów}
TODO klasyfikacja, opisać że są lewo stronnie rekurencyjne etc.

\verb|http://en.wikipedia.org/wiki/Recursive_descent_parser|

\verb|http://en.wikipedia.org/wiki/Left_recursion|
\verb|http://en.wikipedia.org/wiki/Parser_combinator|

\verb|http://stackoverflow.com/questions/17840/how-can-i-learn-about-parser-combinators|

\subsection{Fragmenty implementacji}
Przedstawić fragmenty parsera. Najlepiej go dać jako dodatek jednak w całości.

\begin{lstlisting}
def messageTypeDef: Parser[ProtoMessageType] = opt(comment) 
                                           ~ messageNameDef
                                           ~ meddageBody ^^ {
  case maybeDoc ~ id ~ allFields =>
    // utworzenie instancji ProtoMessageType
  }
\end{lstlisting}


%---------------------------------------------------------------------------
\section{Verifier}
Generalny opis dlaczego musiał powstac

Przedstawić jakie sprawdzania obsługuję. 
Pokazać że obsługuję importy oraz udowodnić dlaczego konieczny jest dodatkowy krok na to.
No bo bez takiego kroku nie wiedziałbym czy przypadkiem gdzies indziej nie zostal zdefiniowany jakis message etc.

\subsection{Obsługiwane weryfikacje}
Moze sub sectiony o tych checkach oraz konkretne przykłady błędów i jak są komunikowane?


%---------------------------------------------------------------------------
\section{CodeGenerator}
Generator kodu w tym przypadku jest bardzo prostą serią transformacji.
Opisać, wspomnieć że korzystam z mustache etc.

\subsection{Język szablonów - Mustache}
Podczas generowania plików HTML klasa \verb|ProtoDocTemplateEngine|

%---------------------------------------------------------------------------
\section{Przygotowanie środowiska do rozwoju ProtoDoc}
\label{sec:prepare_env}
W rozdziale ten pokrótce przedstawię jak należy przygotować środowisko programistyczne celem rozwijania narzędzia \textit{ProtoDoc},
może się to okazać przydatne w przypadku chęci sprawdzenia testów jednostkowych bądź wprowadzenia nowych funkcjonalności do aplikacji.

\subsection{}

\section{Instalacja narzędzia SBT}
ProtoDoc budowany oraz testowany jest przy wykorzystaniu najpopularniejszego obecnie narzędzia do zarządzania buildem w świecie programistów Scala:
Simple Build Tool, w śkrócie zwanym SBT \footnote{SBT - Strona domowa projektu: \href{https://github.com/harrah/xsbt}{https://github.com/harrah/xsbt}}.




%---------------------------------------------------------------------------
\newpage
\chapter{Zrzuty ekranu wygenerowanej dokumentacji}
\label{sec:screenshots}
Rozdział ten zawiera przykładowe zrzuty ekranu wygenerowanych przy pomocy \textit{ProtoDoc} dokumentacji dla różnych typów wiadomości.

\begin{figure}[hc]
 \begin{center}
  \includegraphics[width=\textwidth]{../protodoc_main.png}
  % protodoc_main.png: 949x970 pixel, 93dpi, 25.92x26.50 cm, bb=0 0 735 751
 \end{center}
 \caption{Widok wygenerowanej strony dla typu \textbf{message}}
 \label{msg_page}
\end{figure}

Na Rysunku \ref{msg_page} przedstawiona została strona wygenerowana na podstawie 

\begin{figure}[hc]
 \begin{center}
  \includegraphics[width=\textwidth]{../protodoc_enum.png}
  % protodoc_enum.png: 949x970 pixel, 93dpi, 25.92x26.50 cm, bb=0 0 735 751
 \end{center}
 \caption{Widok wygenerowanej strony dla typu \textbf{enum}}
 \label{enum_page}
\end{figure}


%===========================================================================
\chapter{Rola Testów oraz TDD w procesie tworzenia aplikacji}
\label{chapter:tdd}

Projekt prowadzony był zgodnie z zasadami Test Drivem Development (zwanego dalej \textit{TDD}),
co znacznie ułatwiło ustabilizowanie API oraz głównych konceptów jeszcze we wczesnych etapach tworzenia aplikacji.
Ponad to, metodyka ta umożliwiła pracę z dotychczas nieznanym mi API bez obaw o zniszczenie zaimplementowanych wcześniej funkcjonalności.

Metodykę \textit{TDD} możnaby opisać jako cylk składający się z trzech faz:
\begin{itemize}
 \item napisanie najpierw \small{(sic!)} testu, sprawdzającego automatycznie czy stawiane przed nami oczekiwanie zostało spełnione
 \subitem pewną sub-fazą jest upewnienie się że test faktycznie na stan obecny aplikacji nie przechodzi. Najlepiej aby wiadomość niepowodzenia
          jasno wskazywała na to co jest przyczyną problemu. Jest to istotne nie tyle teraz, podczas implementacji, jednak podczas dalszego rozwoju aplikacji,
          kiedy to być może sprawimy, że ten test przestanie przechodzić - wówczas, \textit{,,kilka tygodni później''}, pomocny komunikat o przyczynie problemu 
          znacznie przyśpieszy zlokalizowanie oraz naprawienie problemu.
 \item implementacji funkcjonalności, tak aby warunki w teście zostały spełnione.
  \subitem należy pamiętać aby była to implementacja minimalna - nie wolno wychodzić ,,do przodu'' z implementacją, nawet jeżeli uważa się,
           że pewna funkcjonalność \textit{prawdopodobnie} będzie niebawem implementowana.
 \item oraz refaktoringu właśnie zaimplementowanych komponentów aplikacji, lub zauważonych podczas implementacji ewentualnych powtórzeń kodu itp.
\end{itemize}

Fazy te w literaturze znane są jako ,,Red - Green - Refactor'', i obrazuje się ją przy pomocy przedstawionego na Rysunku \ref{tdd_cycle} grafu.

\begin{figure}[ch]
 \begin{center}
  \includegraphics[scale=0.8]{tdd_cycle}
 \end{center}
 \label{tdd_cycle}
 \caption{Schemat obrazujący fazy pracy w metodyce \textit{TDD} \small{(źródło: własne)}}
\end{figure}


Przedstawiony powyżej cykl zazwyczaj trwa pomiędzy kilkoma a trzydziestoma minutami. Technika ta jest ściśle związana z samo-dyscypliną programisty
i stosunkowo trudna do zastosowania w przypadku nie stosowania jej na codzień - jednak resultaty, pod postacią wzrostu jakości kodu oraz zmniejszeniu 
czasu traconego na poszukiwania błędów są znaczne.

Oprócz pisania testu zanim powstanie jakakolwiek implementacja, bardzo ważnym elementem fazy implementacji jest jest aby jej celem 
było napisanie \textit{minimalnej ilości kodu doprowadzając test to ,,przejścia''} (spełniania wymagań w nim stawianych). Przykładowo, nie dozwolone jest
implementowanie dodatkowych funkcjonalności (,,na zapas''), nawet jeżeli uważa się iż będą niebawem konieczne podczas fazy implementacji związanej 
z właśnie napisanym testem. Faza implementacji nie może zostać zakończona w przypadku uszkodzenia (sprawienia że inny niż obecnie rozwijany test ,,nie przejdzie'')

\section{Zaimplementowane specyfikacje}
\begin{verbatim}
MustacheFilenameTest:
- Should create mustache template filenames

TagVerifierTest:
validateTags 
- should detect duplicated tags
  + Given an message with duplicated field tags 
  + When tags are validated 
  + Then it should detect duplicates 
  + And errors are about the 'second' and 'fail' fields 
validateTags 
- should should 'OK' a valid tags list
  + Given a valid message 
  + When tags are validated 
  + Then it have not detected any problems 

InnerMessagesTest:
Inner message 
- should be parsed properly
PackageTest:
Package name 
- should be read from proto file with it
InnerInnerMsg package 
- should contain it's super Messages in package name

MultipleProtoFilesTest:
Parser given multiple files 
- should parse multiple seperate (independent) files
CommentsTest:
Comment on top level message 
- should be parsed properly
Comment on field 
- should be parsed properly
- should be parsed properly, even if inline
- should be parsed properly, using JavaDoc style markers
- should be parsed properly, even if spanning multiple lines
Multi Line Comment on top level message 
- should be parsed properly
Multi Line Comment on inner enum 
- should be parsed properly
Comment on enum value 
- should be parsed properly

ProtoBufVerifierTest:
The Verifier should validate field types 
- should detect an unresolvable field
  + Given a message with an invalid fieldtype 
  + When the message is parsed and verified 
  + Then the Verifier report it as invalid 
  + And it should point out that the UnknownType is unresolvable 
- should have no problems with resolvable field Type
  + Given a message with valid, resolvable fieldtype, defined before the message 
  + When the message is parsed and verified 
  + Then the result should contain one HasResolvableField message 
  + And the field should be resolved to the propper type 

RealSimpleParsingTest:
Parsing of an real message, with outer enum 
- should be parsed properly
  + Given A real proto file 
  + When it is parsed 
  + And it is verified 
  + Then parsed size should be 2 
  + And the inner message should be detected 
  + And the inner message should be named properly 
  + And the enum field should have the proper type resolved 
  + And it's tag should be equal 3 
  + And it's resolved type should be the outer enumeration 
  + And the outer enum should be parsed and named properly 

ProtoBufParserTest:
Parser 
- should parse single simple message
- should parse single message with enum
- should have no problems with field modifiers
addOuterMessageInfo 
- should fix package info of inner enums/msgs
  + Will fix packages of: List(ProtoMessageType [InnerMessage] in package: []) 
  + Fix resulted in: List(ProtoMessageType [InnerMessage] in package: [pl.project13.Outer]) 

DeprecationTest:
Message with deprecations 
- the deprecated fields should be detected
- should not detect deprecations where there are none
- should detect deprecation on message type
- should detect deprecation on enum type
- should detect inner types

EnumsTest:
Enum 
- should be parseable inside of an Message
- should be usable as field type
- should be usable even before it's type declaration
- should detect an unresolvable enum or message type reference
Undefined enum 
- should be type checked, so an not existing enum type used as field type will fail compiling

MultipleMessagesInOneFileTest:
Parser 
- should deal with multiple messages defined in the root level of one file
  + Given a proto file with two root devel messages 
  + When the messages are parsed and verified 
  + Then the result should contain two messages 
- should deal with multiple enums and messages defined in root scope, in one file
  + Given a proto file with two root devel messages 
  + When the messages and enum are parsed and verified 
  + Then the result should contain two messages and one enum 

FieldsTest:
Message with 2 fields 
- should in fact have 2 fields
Parser 
- should parse single int32 field
- should parse single fixed32 field
- should parse single sfixed64 field
- should parse single int64 field
- should parse single fixed64 field
- should parse single optional string field
- should parse single required string field with default value

MessageTemplateTest:
ProtoDocTemplateEngine 
- should render simple message page

TableOfContentsTest:
ProtoDocTemplateEngine 
- should render table of contents from sample data

FullIntegrationTest:
ProtoDoc 
- should not fail for simple proto files
  + Given the simple/ director, with proto files 
  + And a valid destination directory 
  + When the files are parsed 
  + Then no exception should be thrown 
  + And the output should be a valid doc 
Passed: : Total 45, Failed 0, Errors 0, Passed 45, Skipped 0

\end{verbatim}


%===========================================================================
\chapter{Zastosowanie ProtoDoc do automatyzacji dokumentacji projektów}

%===========================================================================
\chapter{Spis obsługiwanych funkcjonalności}
Podajemy takie wejście:
\begin{verbatim}
message Person {
  required int32 id = 1;
  required string name = 2;
  optional string email = 3;
}
\end{verbatim}

Następnie wykonanie:

A ostatecznie otrzymujemy taką stronę: \verb|http://protodoc.project13.pl/sample|.

%TODO Tutaj screeny gotowego 


%---------------------------------------------------------------------------------------------------------------------------------------------------------------------------

%---------------------------------------------------------------------------
\appendix
\chapter{Google Protocol Buffers}
\label{cha:appendixA}
W tym dodatku zostanie omówiona idea oraz szczegóły implementacyjne stojące za Google Protocol Buffers, dalej zwanym \textit{ProtoBuf}.
Omówione zostaną również przykładowe zastosowania, oraz dlaczego warto porzucić w niektórych sytuacjach komunikację przy pomocy
XML bądź JSON, na rzecz Protocol Buffers lub mu podobnym binarnym protokołom komunikacji.

%---------------------------------------------------------------------------
\section{Zasada działania}
\label{sec:protobuf_history}

Protocol Buffers powstało pierwotnie jako wewnętrzny projekt Google, mające na celu zmniejszenie ilości ruchu
na łączach oraz zwiększenia szybkości z jaką wiadomości wysyłane między serwerami są serializowane oraz deserializowane.
W momencie pisania wewnątrz Google znajduje się 48,162 typów wiadomości, rozproszonych w 12,183 plikach -- są to imponujące liczby,
obrazujące jak ważnym elementem infrastruktury Google stał się ProtoBuf. \cite{protobuf}

Protocol Buffers, dalej zwane ProtoBuf, opierają się w dużej mierze o język definicji interfejsu (\textit{IDL}) o tej samej nazwie.
Przy jego pomocy definiuje się tak zwane wiadomości, które mogą zawierać pola lub zagnieżdżone wiadomości i/lub enumeracje.
Tak przygotowany opis wiadomości, zapisany w tak zwanym proto-pliku (,,\textit{proto-file}'') następnie poddawany jest kompilacji,
przy pomocy narzędzia \textit{protoc} (\textit{Proto}col Buffers \textit{C}ompiler) czego wynikiem są klasy (o ile taka abstrakcja jest dostępna
w wybranym języku docelowym) które udostępniają bezpieczne (co do typów) API do manipulacji tymi wiadomościami oraz, co ważniejsze, 
,,potrafią się same przeczytać z strumienia bajtów''. Pod tym zwrotem kryje się zwyczajna implementacja fabryki, potrafiąca dekodować 
strumień bajtów zakodowany dla pewnej wiadomości, do postaci obiektu danej klasy.

\newpage
Jako przykład możemy spojrzeć na poniższą deklarację wiadomości (Listing \ref{proto_simple_sam}):
\begin{lstlisting}[label={proto_simple_sam}]
message Person {
  required string name = 1;
  optional string email = 3;
  message PhoneNumber {
    required string number = 1;
    optional PhoneType type = 2 [default = HOME];
  }
  repeated PhoneNumber phone = 4;
}
\end{lstlisting}

Który po skompilowaniu do języka \textit{C++}, wygenerowałby klasę potrafiącą obsłużyć następujące operacje:

\begin{lstlisting}[caption={Przykład klasy ProtoBuf w C++, pisanej na plik}]
Person person;
person.set_name("John Doe");
person.set_id(1234);
person.set_email("jdoe@example.com");
fstream output("myfile", ios::out | ios::binary);
person.SerializeToOstream(&output);
\end{lstlisting}

A następnie możliwe byłoby dokonanie opisywanej wcześniej deserializacji z utworzonej właśnie tablicy bajtów:

\begin{lstlisting}[caption={Przykład klasy ProtoBuf w C++, odczytującej ,,się'' z zapisanego wcześniej binarnego pliku}]
fstream input("myfile", ios::in | ios::binary);
Person person;
person.ParseFromIstream(&input);
cout << "Name: " << person.name() << endl;
cout << "E-mail: " << person.email() << endl;
\end{lstlisting}

Analogicznie wynegerowane klasy można otrzymać w praktycznie dowolnym języku programowania, 
przy czym wspieranymi przez Google językami są C++, Java oraz Python. Do generowania klas dla pozostałych języków
służą implementowane przez społeczność open source pluginy do kompilatora protoc.

Łatwo jest sobie wyobrazić powyższy przykład, piszący zamiast do pliku, na strumień będący odbieranym przez klienta po drugiej stronie.
Tak właśnie w dużym przybliżeniu implementowane są web serwisy przy wykorzystaniu Protocol Buffers. Istnieją dodatkowe pomocnicze fabryki oraz metody
korzystania z ProtoBuf jako zasobów sieciowych, na przykład implementacje Protocol Buffers w oparciu o standard JAX-RS \cite{jaxrs}.


%---------------------------------------------------------------------------
\section{Wyjaśnienie znaczenia pól Protocol Buffers}
\label{sec:typeQialifiers}

\begin{lstlisting}[caption={Przypomnienie wyglądu deklaracji pola wiadomości}]
 required string name = 45 [ default = "sample" ];
\end{lstlisting}

W przedstawionej w 1 sekcji tego rozdziału wiadomości (Listing \ref{proto_simple_sam}) można zauważyć iż każde pole składa się z następujących elementów:

\begin{itemize}
 \item \textbf{Kwalifikatora}, który może przyjąć wartości:
  \subitem \textbf{required} - oznaczającym że pole \textit{musi} zostać ustawione. Generalnie odradza się jego stosowania, ponieważ 
                               pole raz oznaczone tym kwalifikatorem, musi zawsze pozostać required. Nawet jeżeli przestanie być kiedyś w przyszłości używane.
  \subitem \textbf{optional} - oznaczającym że pole może ale nie musi zostać ustawione. Jest to zalecany kwalifikator w większości przypadków.
  \subitem \textbf{repeated} - oznaczającym listę. Pole oznaczone tym kwalifikatorem może powtarzać się dowolną ilość razy w wiadomości.
 \item \textbf{Typu pola} - mogącym być jednym z predefiniowanych w Protocol Buffers typów (wylistowanych w Tabeli \ref{protoYypes_all}, lub typem wiadomości lub enumeracji
                            zdefiniowanym przez użytkownika.
 \item \textbf{Nazwy pola} - nazwa pola, poprawne indentyfikatory są zgodne z standardem stosowanym przez Javę
 \item \textbf{TAG}u - tag jest specjalną liczbą, nie mogącą powtarzać się w obrębie jednej wiadomości. Liczba ta jest wykorzystywana podczas kodowania 
                       wiadomości do jej możliwie najoptymalniejszej postaci. Warto pamiętać że umieszczenie często używanych pól jako pierwszych \textit{ma wpływ}
                       na ostateczny rozmiar wiadomości. Ponieważ w przypadku pól o wysokich tagach, które rzadko, lub nigdy nie są ustawiane, nie zostaną one w ogóle 
                       umieszczone w zserializowanej postaci wiadomości - oszczędzając cenne bajty.
 \item \textbf{default} - wartość domyślna \textit{nie jest konieczna} jednak możliwa do podania przy każdym polu -- sens ma oczywiście jednak tylko w przypadku 
                          podania jej dla pola będącego typu \textbf{optional}, ponieważ w przypadku \textbf{required} wartość ta i tak zostałaby nadpisana.
                          W przypadku pola typu enumeracyjnego, możliwe jest podanie stałej z wewnątrz tej enumeracji jako wartości domyślnej.
\end{itemize}

\begin{table}
\begin{center}
\begin{tabular}{llll}
sint32 & bytes & double & uint64\\
sint64 & string & float & uint32\\
fixed32 & bool & int32 & int64 \\
sfixed32 & sfixed64 & fixed64 &  \\
\end{tabular}
\end{center}
\label{protoTupes_all}
\caption{Predefiniowane typy w ProtoBuf}
\end{table}

Poza wspomnianymi predefiniowanymi typami, pola mogą oczywiście mieć typ innej wiadomości. Tym sposobem można modelować nawet skomplikowane modele domenowe
w obrębie wiadomości ProtoBuf. Innym typem który mogą przyjąć pola, są type enumeracyjne. Deklaruje się je w podobny sposób jak wiadomości, jednak 
ich pola mają mniej elementów koniecznych do podania. Przykład deklaracji enumeracji można zobaczć na Listingu \ref{lst:enumDeclaration}.

\begin{lstlisting}[caption={Deklaracja oraz zastosowanie typu enumeracyjnego}, label={lst:enumDeclaration}]
enum TheEnum {
  SMALL = 1;
  LARGE = 2;
}

message It {
  optional TheEnum itsName = 1 [ default = LARGE ]
}
\end{lstlisting}

Jak widać, możliwe jest również podanie nazwy \verb|LARGE| jako wartości domyślnej pola.


%---------------------------------------------------------------------------
\section{Zestawienie wydajności mechanizmów serializacji na JVM}
\label{sec:serialization_speed}

Poniewać w efekcie, ProtoBuf służy efektywnej serializacji i deserializacji obiektów, bardzo ważna jest jego wydajność na realnych obiektach.
Testy takie przeprowadzono w ramach projektu umieszczonego na githubie pod adresem: \href{https://github.com/eishay/jvm-serializers/wiki/}{https://github.com/eishay/jvm-serializers/wiki/}.
Benchmark \cite{proto_benchmark} ten został wykonany w roku 2011, oraz uwzględnia najważniejsze metody serializacji danych na JVMie, takie jak
natywna serializacja Java, serializacja danych do JSON przy pomocy różnych bibliotek oraz oczywiście ProtoBuf oraz jego konkurencji - Thrift.

\textbf{Serializacja + Deserializacja:}\\
Czas spędzony na serializacji oraz deserializacji tego samego obiektu:

\begin{figure}[ch!]
\includegraphics[width=\textwidth]{proto_1}
% \includegraphics[width=\textwidth]{proto_2} TODO TODO TODO
\end{figure}

ProtoStuff, alternatywna implementacja protokołu ProtoBuf wiedzie prym w tym benchmarku, jednak protobuf nadal jest stanowczno wiadącą implementacją.
Najważniejszą obserwacją tutaj jest jednak porównanie binarnych protokołów, które zmieściły się na powyższym wykresie, do metod serializacji JSONowej.
Dla przykłądu Google Gson, jedna z popularniejszych bibliotek serializacji Java -> JSON, tą samą serializację przeprowadziła w \textbf{25778ns}, co w porównaniu
z czasem osiągniętym przez ProtoBuf, równym \textbf{6132ns} jasno wyraża przewagę z jaką ProtoBuf wyprzedza serializację ,,klasyczną''.

\newpage
\textbf{Rozmiar zserializowanej wiadomości:}\\
Rozmiar liczony w bajtach wiadomości po zserializowaniu:

\begin{figure}[ch!]
\includegraphics[width=\textwidth]{serialized_size_1}
% \includegraphics[width=\textwidth]{proto_2} TODO TODO TODO
\end{figure}

Przy porównywaniu rozmiaru zserializowanego obiektu ponownie protokoły binarne, z ProtoBuf na czele wiodą prym... 
Konkurencja typu serializacji Javowej lub JSON niestety nie ma tutaj wiele do powiedzenia, powodem jest iż Java serializując obiektu musi
zatrzymać wszystkie nazwy klas aby potrafić się do nich odwołać podczas deserializacji; W przypadku JSONa, mamy do czynienia z dużą ilością nadmiarowych danych
w postaci reprezentowania liczb jako tekst, oraz dużej ilości nadmiarowych znaków w postaci \verb|{| oraz \verb|}| itp.



%---------------------------------------------------------------------------
\newpage
\section{Przykładowe definicje wiadomości}
\label{sec:proto_file_examples}

Poniżej zostanie przedstawione kilka przykładowych definicji wiadomości, celem zapoznania czytelnika z formatem i składnią protobuf w
sposób bardziej praktyczny (wizualny).

\begin{lstlisting}[caption={Definicja wiadomości zawierająca enum oraz wartości domyślne}]
message SearchRequest {
  required string query = 1;
  optional int32 page_number = 2;
  optional int32 result_per_page = 3 [default = 10];

  enum Corpus {
    UNIVERSAL = 0;
    WEB = 1;
    IMAGES = 2;
    LOCAL = 3;
    NEWS = 4;
    PRODUCTS = 5;
    VIDEO = 6;
  }

  optional Corpus corpus = 4 [default = UNIVERSAL];
}
\end{lstlisting}

\begin{lstlisting}[caption={Definicja wiadoomści wykorzystującej inną wiadomość}]
message SearchResponse {
  repeated Result result = 1;
}

message Result {
  required string url = 1;
  optional string title = 2;
  repeated string snippets = 3;
}
\end{lstlisting}

\newpage
\begin{lstlisting}[caption={Definicja wiadomości korzystającej z wewnętrznej wiadomości}]
package pl.project13.protodoc;

message SearchResponse {
  message Result {
    required string url = 1;
    optional string title = 2;
    repeated string snippets = 3;
  }
  repeated Result result = 1;
}
\end{lstlisting}

Więcej przykładów można zobaczyć na stronie domowej Protocol Buffers - \href{http://code.google.com/intl/pl-PL/apis/protocolbuffers/docs/proto.html}{http://code.google.com/intl/pl-PL/apis/protocolbuffers/docs/proto.html}.
Niestety podczas realizacji projektu nie udało się pokryć pełnego formatu protocol buffers, jest to jednak logiczny kierunek dalszego rozwoju aplikacji.

\chapter{Podstawy języka Scala oraz Scala Parser Combinators}
\label{cha:appendixB}
Celem tego dodatku jest przybliżenie czytelnikowi języka ,,Scala'' aby w wystarczająco płynny sposób mógł czytać przykłady kodu używane w tym dokumencie.

%---------------------------------------------------------------------------
\section{Krótka historia języka}
\label{sec:scala_history}
Język Scala (,,Scalable Language'') najłatwiej jest przedstawić jako hybrydę dwóch znanych nurtów programowania: programowania obiektowego oraz funkcyjnego, wraz z 
powiązanymi z nimi językami programowania. Twórca języka Scala, Martin Oderski \footnote{Martin Odersky - Strona domowa: \href{http://lamp.epfl.ch/~odersky/}{http://lamp.epfl.ch/~odersky/}}
był ściśle związany z językiem Java - był głównym projektantem generyków w Javie (\textit{Java Generics}) oraz głównym autorem utrzymywanej po dziś dzień
serii kompilatorów \textbf{javac} \cite{OderskyWywiad}.

Jako konkretnych ,,rodziców'' można by wskazać: 
\begin{itemize}
 \item \textbf{Java} - jako reprezentant nurtu obiektowego 
 \item oraz języki: \textbf{Haskell}, \textbf{SML} oraz pewne elementy języka \textbf{Erlang} (głównie \textit{Actor model}).
\end{itemize}

O języku Scala można myśleć jako połączeniu tych nurtów. Dostępne są klasyczne elementy języków funcyjnych,
takie jak pattern matching czy nacisk na immutability wszystkich tworzonych obiektów. Jest to zwłaszczw widoczne w domyślnych implementacjach
kolekcji, których nie można modyfikować - a tworzy się za każdym razem nową listę, współdzieląc między nimi elementy które są niezmienne.

%---------------------------------------------------------------------------
\section{Podstawy}
\label{sec:scala_basics}
Ta sekcja służy przybliżeniu czyletnikowi języka \textit{Scala} na poziomie wystatczającym aby swobodnie czytać przykłady
kodu umieszczone w tej pracy. W niektórych przykładach pomijane są przypadki skrajne lub nietypowe, celem szybkiego oraz 
jasnego przedstawienia minimum wiedzy na temat języka aby móc swobodnie go ,,czytać''.

\textit{Scala} jest językiem statycznie typowanym posiadającym lokalne ,,Type Inferrence''. Pozwala to kompilatorowi 
\textit{scalac} na ,,odnajdywanie'' typów wszystkich zmiennych oraz typów zwracanych przez metody podczas kompilacji,
bez potrzeby definiowania ich wprost (poza konkretnymi wyjątkami np. funkcji wprost rekurencyjnych).

Użycie nawiasów \verb|()|, średnika \verb|;| oraz kropki \verb|.| jest analogiczne jak w przypadku Javy, 
jednak w wielu przypadkach opcjonalne gdyż kompilator jest w stanie wydedukować gdzie powinny się znaleźć.

\begin{lstlisting}
 val value: Int = Option(42);
 val other: Int  = value.orElse(0);

 // moze zostac zastąpione
 val value = Option(42)
 val other = value orElse 0 // "infix notation"
\end{lstlisting}

Jednym z ciekawych przykładów stosowania notacji bez nawiasów i kropek jest \textit{ScalaTest}\footnote{ScalaTest - framework do testowania - http://www.scalatest.org}
(przy którego pomocy pisano testy w tym projekcie). Przykładowa \textit{asercja} napisana w \textit{DSL}u definiowanym przez tą bibliotekę wygląda następująco:
\begin{lstlisting}
 messages should (contain key ("Has") and not contain value ("NoSuchMsg"))
\end{lstlisting}

Dostępne jest wiele sposobów definiowania metod / pól w klasie,
w efekcie (na poziomie bytecode), wszystkie przekładane są na wywołania metod. Dostępne są słowa kluczowe:
\begin{itemize}
 \item \textbf{def}, \textit{def}iniujący zwyczajną metodę instancyjną. Warto nadmienić że Javowa koncepcja pojęcia \textit{static} nie jest dostępna z poziomu Scala.
 \item \textbf{val}, deklarujący ,,stałą'' - to jest metodę która raz zawołana, zwróci wartość oraz pole to będzie konsekwentnie zwracać tą samą wartość. 
              Dodatkowym efektem jest traktowanie zmiennych tego typu analogicznie do Javowych zmiennych z modyfikatorem \textbf{final}.
 \item \textbf{var}, deklaruje zwyczajną ,,zmienną'', do jakiej przyzwyczajeni jesteśmy z Java.
 \item modyfikator \textbf{lazy}, wpływający na moment inicjalizacji zmiennej - metody zadeklarowane z modyfikatorem \textbf{lazy} 
       zostaną dopiero zainicjalizowane podczas pierwszego odwołania się do tego pola z innego miejsca w kodzie. 
       W przypadku pary \textbf{lazy val}, metoda ta zostanie zawołana jedynie jednokrotnie, a zwrócona po raz pierwszy wartość
       zostanie zapisana w cache oraz będzie konsekwentnie zwracana podczas ponownych wywołań tej metody.
\end{itemize}

Modyfikator lazy pozwala na budowę eleganckich konstrukcji z jakimi mamy do czynienia w przypadku na przykład Parser Combinators, omówionych szczegółowo w kolejnych sekcjach.


%---------------------------------------------------------------------------
\section{Traits - wmieszanie zachowania do klasy}
\label{sec:traits}

Słowo kluczowe \textbf{trait} rozpoczyna definicję typu zwanego traitem.
Implementacja nie różni się (na potrzeby tego szybkiego omówienia) od implementowania klasy,
jednak różnica jest podczas ,,dziedziczenia'' przy wykorzystaniu traitów. Nie mówimy bowiem o ,,dziedziczeniu'' 
w przypadku \textit{trait}ów, a o ,,wmieszaniu'' (ang. \textit{mixin} - wmieszanie) zachowania do klasy konkretnej.

Poniżej został przedstawiony najprostrzy trait zawierający jakieś zachowanie, oraz jeden ze sposobów jego wmieszania do klasy konkretnej.
Warto zauważyć że w przypadku wmieszania \textit{trait}a \verb|A| do klasy \verb|Test|, wprowadzamy między nimi relację ,,Test \textbf{IS-A} A'',
analogicznie jak w przypadku dziedziczenia.

\begin{lstlisting}
trait A { 
  def test = "A" // definicja metody zwracającej "A"
}

class Test extends A { } // wmieszanie A

(new Test).test // skompiluje i wykona sie poprawnie
\end{lstlisting}

Co ciekawe, nie zauważamy różnicy w przypadku składni odnoszącej się do dziedziczenia dwóch klas konkretnych, oraz wmieszania traita.
Składnia ulega zmianie w przypadku korzystania z więcej niż jeden trait lub domieszania traita do klasy która już dziedziczy po innej klasie,
wówczas zamiast słowa kluczowego \textbf{extends} należy stosować \textbf{with} (nie dozwolone jest wielokrotne zapisanie \textbf{extends},
jednak wielokrotne \textbf{with} są często spotykane). Przykład wmieszania większej ilości traitów zostanie przedstawiony poniżej.

Jest to namiastka dziedziczenia wielobazowego jednak Scala dzięki swojemu bardzo rygorystycznemu kompilatorowi jest w stanie 
uniknąć sytuacji gdzie dziedziczenie wielobazowe byłoby niebezpieczne (klasyczne przykłady 
problematycznych sytuacji w przypadku dziedziczenia wielobazowago można przeczytać w ,,Symfonii C++'', autorstwa pana Grębosza \cite{symfonia}).

Kompilator \textit{scalac} przy napotkaniu konflintów nazw mogących doprowadzić do niejasności ,,którą metodę należy zawołać'', nie skompiluje takiego kodu
oraz poprosi o rozwiązanie konfliktu w sposób explicite. Jako przykład rozważmy dwa \textit{trait}y udostępniające metodę \verb|def test: String|:

\begin{lstlisting}
trait A { def test = "A" }
trait B { def test = "B" }

class Example extends A with B {
  // blad kompilacji!
}
\end{lstlisting}

Przy napotkaniu problemu tego typu kompilator zgłosi:
\begin{verbatim}
error: overriding method test in trait A of type => java.lang.String;
                  method test in trait B of type => java.lang.String needs `override' modifier
class Example extends A with B {
\end{verbatim}

Dzieje się tak ponieważ \textbf{scalac} próbuje odnaleźć która metoda powinna mieć większą wagę, a tymsamym powinna zostać wywołana.
Ponieważ nie jesteśmy w stanie dodać modyfikatora \textbf{override} do żadnego z \textit{trait}ów (ponieważ nie nadpisują one tej metody, a jedynie deklarują),
jedynym możliwym miejscem na rozwiązanie tego konfliktu jest uzupełnienie \verb|Example| o następujący fragment kodu, rutujący poprawnie nasze wywołanie metody:

\begin{lstlisting}
class Example extends A with B {
 // selektywne odwolanie sie do metody konkretnego supertypu
 override def test = super[B].test
}

new Example().test // poprawne
\end{lstlisting}

%---------------------------------------------------------------------------
\section{Case Class oraz Pattern Matching}
\label{sec:caseclass}

Klasy deklarowane przy pomocy \verb|case class| niosą ze sobą pewne ułatwienia, które generuje za nas kompilator.
Case klasy są wykorzystywane w bardzo wielu miejscach w ProtoDoc, ze względu na ich znaczną zwięzłość zapisu oraz wygodną współpracę z konstrukcją
\verb|match|, która zostanie za moment wyjaśniona.

Case klasę \small{(pozwolę sobie przyjąć taki, mieszający dwa języki, sposób nazywania jej, z racji problematycznego sesownego przetłumaczenia słowa case 
(ang. przypadek) w tym zwrocie)} definiujemy przy pomocy konstrukcji przedstawionej na Listingu \ref{lst:case_class}.

\begin{lstlisting}[caption={Deklaracja case klasy},label={lst:case_class}]
case class SampleProtoField(name: String, value: Long)
\end{lstlisting}

oraz odpowiada w przybliżeniu implementacji przedstawionej poniżej, na Listingu \ref{lst:case_class_by_hand}:

\begin{lstlisting}[caption={Ręczna implementacja case classy},label={lst:case_class_by_hand}]
class SampleProtoField {
 private def this(__name: String) = {
  this()
  _name = __name
 }

 private val _name = null

 def name =  _name

 def equals = /**/
 def hashCode = /**/
 def toString = /**/
}

object SampleProtoField {
 def apply(name: String) = new SampleProtoField(name)
 def unapply(field: SampleProtoField) = field.name
}
\end{lstlisting}

Dzieje się tutaj bardzo dużo ciekawych rzeczy sięgających głęboko po możliwości Scala, jednak w efekcie umożliwia nam:
\begin{itemize}
 \item domyślne utworzenie niezmiennych pól dla każdego argumentu w konstruktorze case klasy (val)
 \item automatyczne wygenerowanie getterów dla tych pól
 \item przyjemną dla oka implementację \verb|toString()|
 \item implementacje \verb|equals()| and \verb|hashCode()| (programiści Java znają ból generowania tych metod ręcznie)
 \item wygenerowanie ,,companion object'' pozwalającego na:
 \subitem tworzenie konstrukcji typu \verb|SampleProtoField("")| zamiast \verb|new SampleProtoField("")| (poprzez implementację apply)
 \subitem korzystanie z tej klasy w konstrukcji match (poprzez implementację unapply)
\end{itemize}

Najciekawsze dla nas są automatyczne implementacja apply / unapply, które w efekcie pozwalają na następujące linie:

\begin{lstlisting}
val it: SampleProtoField = SampleProtoField("name") // apply

val SampleProtoField(name) = "some name"
\end{lstlisting}

A jeżeli sięgnąć po konstrukcję match, pozwala ona na konstrukcje ,,wyjmujące'' wartości pól z skomplikowanych case class:

\begin{lstlisting}
myCaseClass match {
  case SampleProtoField(name) => println(name)
  case ComplicatedProtoField(name, type, _, _) => println(name +" & "+ type)
}
\end{lstlisting}

Konstrukcja \verb|match|, wywodzi się z programowania funkcyjnego. Mowa tutaj o ,,pattern matching''
znanym z chociażby \textbf{Erlanga}. Dzięki tej konstrukcji da się ominąć wiele linii kodu w stylu \verb|if(x) {x = x.getX(); method(x);}|.
Technika ta została zastosowana w bardzo wielu miejscach aplikacji, włącznie z parserem oraz weryfikatorem.


%---------------------------------------------------------------------------
\section{Implicit Conversions - konwersje ,,domniemane''}
\label{sec:implicit}
Scala pomimo że jest językiem silnie statycznie typowanym pozwala na pewne zabiegi aby ułatwić pracę w tak rygorystycznym systemie typów.
Jednym z tych rozwiązań są tak zwane ,,Implicit Conversions'', będące typem metod, które kompilator może próbować zastosować podczas gdy 
potrzebna jest automatyczna konwersja z typu A na B. Najłatwiej będzie omówić to na prostym przykładzie (Listing \ref{lst:simple_implicit_example},
także spójrzmy na poniższe przypisanie liczby typu \verb|scala.Int| do zmiennej typu \verb|java.lang.String|:

\begin{lstlisting}[caption={Przykład wystąpienia implicit conversion}, label={lst:simple_implicit_example}]
val num: Int = 42
val string: String = num // compile time error!
\end{lstlisting}

Przykład przedstawiony na Listingu \ref{lst:simple_implicit_example} \textit{nie skompiluje się}, a kompilator odpowiedziałby następującym komunikatem:

\begin{verbatim}
 <console>:8: error: type mismatch;
 found   : Int
 required: String
       val string: String = num
\end{verbatim}

Dopisanie implicit konwersji w zasięgu widoczności tego przypisania, pozwoli natomiast kompilatorowi zauważyć iż dostępna jest metoda potrafiąca przeprowadzić
konwersję z typu \verb|Int | na \verb|String|, oraz ją zastosować. Implementację takiej konwersji przedstawiono na Listingu \ref{lst:an_implicit}.

\begin{lstlisting}[caption={Implementacja oraz zastosowanie konwersji domniemanej --- \textit{Implicit Conversion}}, label={lst:an_implicit}]
implicit def num2str(num: Int) = num.toString
// == implicit def num2str(num: Int): String = num.toString

val num: Int = 42
val string: String = num // ok!
\end{lstlisting}

To co się rzeczywiście dzieje podczas kompilacji, to zwyczajne wstawienie metody num2str, w linii z przypisaniem liczby do zmiennej typu String,
w następujący sposób: \verb|num2str(num)|. Istnieje więcej szczegółowych zasad dotyczących konwersji domniemanych, na przykład która konwersja powinna zostać 
zastosowana w przypadku większej ilości metod pozwalających na poprawne wykonanie przypisania, jednak nie będziemy w nie wnikać, ponieważ na potrzeby zrozumienia
zastosowanych DSL\footnote{DSL - Domain Specific Language} sama informacja o istnieniu tych konwersji powinna być wystarczająca.




%---------------------------------------------------------------------------
\section{Scala Parser Combinators}
\label{sec:scala_parser_combinators}
W tym rozdziale zostaną pokrótce przedstawione Scala Parser Combinators, oraz ich składnia.
Zostaną poruszone również tematy dotyczące wydajności kombinatorów oraz jak ustrzec się przez zanadto nawracającym się parserem.

Podstatową informacją którą należy sobie przyswoić podczas pracy z kombinatorami parserów, jest idea operowania na funkcjach wyższego rzędu - 
bo dokładnie tym są kombinatory. Najpierw jednak zdefiniujmy najprostszy możliwy parser (przedstawiony na Listingu \ref{lst:super_simple_parser}):

\begin{lstlisting}[caption={Najprostszy możliwy parser}, label={lst:super_simple_parser}]
object SimplestParser extends JavaTokenParsers {
  def a: Parser[String] = "a"
}
\end{lstlisting}

Przedstawiona na listingu imponująca implementacja parsera jednego znaku, de facto dokonywana jest przez implicit konwersję (dostarczoną przez \verb|JavaTokenParsers|), 
z typu \verb|String| na typ \verb|Parser[String]|. Zgodnie z oczekiwaniami ,,parsuje'' on jedynie jeden znak - oraz zwraca samego siebie (wspomniane ,,a'').
W przypadku napisu dłuższego niż samo ,,a'', prymitywny parser możemy zdefiniować jako funkcję przyjmującą napis wejściowy a zwracającą krotkę 
przeprasowanego elementu, oraz pozostałej części napisu. W szczególności Parser zatem można zdefiniować jako:

\begin{center}
\begin{tabular}{|p{\textwidth}|}
\hline
\textbf{Definicja 1} \\
\textit{Parser} zdefiniowany jest jako funkcja przyjmująca napis wejściowy, a zwracająca krotkę sparsowanego elementu oraz pozostałego napisu wejściowego. \\ 
\cite{monadparsing}
\\ \hline
\end{tabular}
\end{center}

~\\\*

Mając tak zdefiniowany parser, możemy przedstawić definicję kombinatora parserów:

\begin{center}
\begin{tabular}{|p{\textwidth}|}
\hline
\textbf{Definicja 2} \\
\textit{Kombinator parserów}, jest \textit{funkcją wyższego rzędu},
która przyjmuje jako argumenty funkcje parsujące, oraz zwraca nową funkcję utworzoną poprzez ich kombinację -- zależną od jego implementacji. \\

\cite{monadparsing}
\\ \hline
\end{tabular}
\end{center}

W przypadku Scali, kombinatory implementowane są poprzez metody wypisane w Tabeli \ref{combinators_table}, zdefiniowane w typie \verb|Parser|.

% /////////////////// tabelka tu byla

Warto tutaj przytoczyć implementację dwóch kombinatorów, ponieważ pokazują jak olbrzymią elastyczność w definiowaniu ich uzyskać
poprzez wykorzystywanie bardzo prostych kombinatorów (wewnętrznie zdefiniowanych \verb\|||\ oraz \verb|&&&|, z których jako programista korzystający
z Scala Parser Combinators raczej korzystać nie będziemy). Warte zauważenia na Listingu \ref{optrep} jest na przykład wywołanie \verb|rep(p)|,
otóż dzięki elastyczności Scali, możliwe jest zaimplementowanie Parserów w ten sposób że do wywołania rep(p) nie koniecznie musi dojść, pomimo że 
mamy do czynienia z zwyczajnymi wywołaniami metod. Umożliwia to tak zwany ,,pass-by-name parameter'', jednak znacznie wykracza to poza zakres konieczny 
do zrozumienia działania Parser Combinators, także pozostaniemy przy stwierdzeniu iż mamy tutaj do czynienia z leniwą ewaluacją tej metody.

\begin{lstlisting}[caption={Implementacja parserów opt i rep}, label={optrep}]
def opt(p: Parser): Parser = p ||| empty;   // p ? p : empty
def rep(p: Parser): Parser = opt(rep1(p));  // p* = [p+]
def rep1(p: Parser): Parser = p &&& rep(p); // p+ = p ~ p*
\end{lstlisting}


Przy odpowiednim zastosowaniu tych prostych zasad, można bardzo łatwo tworzyć parsery nawet skomplikowanych gramatyk.
Niestety z natury działania kombinatorów, możliwe jest wygenerowanie parsera który będzie wykonywał wielokrotne oraz głębokie nawroty.
Celem zmniejszenia miejsc narażenia się na zbyt dużą klasę wygenerowanego parsera \textit{LL(k)}, możemy zastosować kombinator \verb|~!|
który zgłosi błąd jeżeli dana kombinacja nie jest możliwa do rozwiązania w sposób generujący Parser klasy \textit{LL(1)} w tym miejscu.
Powinno się stosować ten operator jeżeli tylko możliwe, aby zagwarantować sobie wczesne ostrzeżenie o nie optymalnej formie gramatyki.
Wówczas można przeprowadzić faktoryzację lewostronną, celem wyciągnięcia wspólnego symbolu dla dwóch lub więcej produkcji ,,przed nawias'', zmniejszając klasę parsera.

\begin{lstlisting}[caption={Zastosowanie kombinatora transformującego}, label={so_simple}]
def TRUE: Parser[Boolean] = "true" ^^ { b => true }

//                                      without ^^, 
// def TRUE: Parser[String] = "true" // value is String!

\end{lstlisting}

Ostatnim elementem koniecznym do wprowadzenia zanim przejdziemy do przykładów pełnych implementacji prostych parserów, jest kombinator \verb|^^|,
oznaczający ,,\textit{transformację}''. Kombinator ten służyć nam będzie do produkowania typów które chcemy produkować z danego sparsowanego elementu,
nie natomiast prymitówów takich jak listy i typy podstawowe, które zostałyby zwrócone przez parser bez zastosowania tego kombinatora.
Prostym przykładem jest parser przedstawiony na Listingu \ref{so_simple}, który transformuje napis ,,true'' na wartość typu \verb|Boolean|.


\begin{table}[ch]
\begin{center}
\begin{tabular}{|l|l|p{7cm}|}
\hline
Metoda & Nazwa & Przykład \\ \hline
\verb|~| & sekwencja & \verb|"a" ~ "b"| parsuje ciąg ,,ab'', oraz zwraca ,,ab''\\
\verb|~>| & sekwencja, odrzucając lewą stronę & \verb|,,a'' ~> ,,b''| parsuje ciąg ,,ab'', oraz zwraca ,,b''\\
\verb|<~| & sekwencja, odrzucając prawą stronę & \verb|,,a'' <~ ,,b''| parsuje ciąg ,,ab'', oraz zwraca ,,a''\\
\verb\|\ & alternatywa & ,,a'' | ,,b'' parsuje ciąg ,,a'' lub ,,b'', oraz zwraca ,,a'' lub ,,b''\\
\verb\^^\ & transformacja & \verb|,,a'' ^^ () => Any| parsuje ciąg ,,a'', oraz zwraca do, co zwróci funkcja przekazana jako argument tego kombinatora. Zostanie opisany szczegółowo poniżej. \\
\verb|opt()| & sekwencja & opt(,,b'') parsuje ciąg ,,'' lub ,,b'', oraz zwraca Option[String], mogące zawierać ,,b'' lub nic\\
\verb|rep()| & powtórzenie & rep(,,b'') parsuje ciąg składający się z powtórzeń ciągu ,,b'', oraz zwraca List(,,b'', ,,b'', ,,b'', ...)]\\
\verb|repsep()| & powtórzenie, z separatorami & repsep(,,b'', ,,,'') parsuje ciąg składający się z powtórzeń ciągu ,,b'' oddzielonych znakiem ,,,'', oraz zwraca List(,,b'', ,,b'', ,,b'', ...)]\\
\verb|rep()| & powtórzenie & rep(,,b'') parsuje ciąg składający się z powtórzeń ciągu ,,b'', oraz zwraca List(,,b'', ,,b'', ,,b'', ...)]\\ \hline
\end{tabular}
\end{center}
\caption{Dostępne podstawowe kombinatory parserów}
\label{combinators_table}
\end{table}

\subsection{Przykłady parserów}
Przedstawione zostaną dwa proste przykłady parserów, mieszczące się na 1 stronie A4, oraz realizujące przydatne operacje, 
takie jak parsowanie formatu JSON, oraz obliczenie prostego wyrażenia matematycznego. Przykłady te mogą posłyżuć jako pomoc w unaocznieniu elegancji
oraz potęgi Parser Combinators.

Jako pierwszy przykład parsera chciałbym w tym miejscu zacytować implementację parsera JSON \footnote{JSON - Java Script Object Notation} umieszczonego w książce
\textit{Programming in Scala} autorstwa Martina Oderskiego \cite{odersky_scala}:

\begin{lstlisting}
import scala.util.parsing.combinator._

class JSON extends JavaTokenParsers {   

  def obj: Parser[Map[String, Any]] = 
    "{"~> repsep(member, ",") <~"}" ^^ (Map() ++ _)
    
  def arr: Parser[List[Any]] =
    "["~> repsep(value, ",") <~"]" 

  def member: Parser[(String, Any)] = 
    stringLiteral~":"~value ^^ 
      { case name~":"~value => (name, value) }

  def value: Parser[Any] = (
    obj
  | arr 
  | stringLiteral
  | floatingPointNumber ^^ (_.toDouble) 
  | "null"  ^^ (x => null) 
  | "true"  ^^ (x => true) 
  | "false" ^^ (x => false)
  )
}
\end{lstlisting}

Kod nie dość że jest elegancki, to okazuje się, gramatyka wykorzystana w tym przykładzie (a zatem i wygenerowany parser) jest klasy \textit{LL(1)}.
Oznacza to, że kosztowny mechanizm nawrotów nigdy nie zostanie wykorzystany podczas parsowania JSONa przy pomocy powyższego parsera. Jest to ciekawy argument
przeciwko zarzutowi stawianemu kombinatorom parserów, iż generują bardzo powolne (nawracające się) parsery -- jest to oczywiście nie prawda, ponieważ to 
od zastosowanej gramatyki zależy jakiej klasy otrzymamy w efekcie parser.

~\\\*

Drugim przykładem parsera jest zaimplementowany przeze mnie celem pokazania kombinatorów parserów prosty parser,
który w efekcie działania oblicza b. proste wyrażenia matematyczne. Parser ten można raczej traktować jako ciekawostkę niż pełną implementację Idei, 
jednak obrazuje on bardzo dobrze jak wykorzystywane są operacje \verb|^^| celem wpasowania parsowanego drzewa do własnego modelu -- analogicznie
jak miało to miejsce podczas implementowania parsera ProtoDoc.

\begin{lstlisting}
object MathParser extends RegexParsers with ImplicitConversions {
  def number: Parser[Long] = "[0-9]+".r ^^ { _.toLong }
  def num: Parser[Long] = lp ~> number <~ rp
  
  def add: Parser[AOperation] = ("+" | "-" ) ^^ { AOperation(_) }
  def mult: Parser[MOperation] = ("*" | "/" ) ^^ { MOperation(_) }

  def op: Parser[Long] = lp ~> num ~! (add | mult) ~! num <~ rp ^^ {
    case num1 ~ op ~ num2 => op.perform(num1, num2)
  }

  def lp: Parser[Option[String]] = opt("(")
  def rp: Parser[Option[String]] = opt(")")
  def stop: Parser[Option[String]] = opt(";")
  
  def parse(string: String): Any = parseAll(op, string) match {
    case Success(res, _) => res
    case e => throw new RuntimeException(e.toString)
  }
}

object MathParserTest extends App {
  override def main(args: Array[String]) {
    import MathParser._
    println(parse("10+15")) // 25
    println(parse("10*2")) // 20
  }
}
\end{lstlisting}


Kod ponownie jest zwięzły, czytelny oraz scentralizowany w jednym miejscu. W ramach przypomnienia jak podobny parser wyglądałby w GNU Bison, 
można spojrzeć tutaj: \href{http://www.usna.edu/Users/cs/lmcdowel/courses/si413/F10/labs/L04/calc1/ex1.html}{http://www.usna.edu/Users/cs/lmcdowel/courses/si413/F10/labs/L04/calc1/ex1.html}
gdzie umieszczony został przykład podobnego parsera. Wadami rozwiązania problemy przy pomocy Bison oraz Flex jest korzystanie z wielu narzędzi,
wielu plików, oraz wielokrokowy proces budowania projektu -- co w kontraście do 1 pliku, 1 języka, oraz braku dodatkowych kroków, faktycznie daje Scali pewną przewagę.

%---------------------------------------------------------------------------
\bibliographystyle{alpha}
\bibliography{myrefs}

\end{document}
