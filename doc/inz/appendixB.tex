\chapter{Podstawy języka Scala oraz Scala Parser Combinators}
\label{cha:appendixB}
Celem tego dodatku jest przybliżenie czytelnikowi języka ,,Scala'' aby w wystarczająco płynny sposób mógł czytać przykłady kodu używane w tym dokumencie.

%---------------------------------------------------------------------------
\section{Krótka historia języka}
\label{sec:scala_history}
Język Scala (,,Scalable Language'') najłatwiej jest przedstawić jako hybrydę dwóch znanych nurtów programowania: programowania obiektowego oraz funkcyjnego, wraz z 
powiązanymi z nimi językami programowania. Twórca języka Scala, Martin Oderski \footnote{Martin Odersky - Strona domowa: \href{http://lamp.epfl.ch/~odersky/}{http://lamp.epfl.ch/~odersky/}}
był ściśle związany z językiem Java - był głównym projektantem generyków w Javie (\textit{Java Generics}) oraz głównym autorem utrzymywanej po dziś dzień
serii kompilatorów \textbf{javac} \footnote{Wywiad z Martinem Odersky na temat korzeni języka Scala - \href{http://www.artima.com/scalazine/articles/origins_of_scala.html}{www.artima.com/.../origins\_of\_scala}}.

Jako konkretnych ,,rodziców'' można by wskazać: 
\begin{itemize}
 \item \textbf{Java} - jako reprezentant nurtu obiektowego 
 \item oraz języki: \textbf{Haskell}, \textbf{SML} oraz pewne elementy języka \textbf{Erlang} (głównie \textit{Actor model}).
\end{itemize}

O języku Scala można myśleć jako połączeniu tych nurtów. Dostępne są klasyczne elementy języków funcyjnych,
takie jak pattern matching czy nacisk na immutability wszystkich tworzonych obiektów. Jest to zwłaszczw widoczne w domyślnych implementacjach
kolekcji, których nie można modyfikować - a tworzy się za każdym razem nową listę, współdzieląc między nimi elementy które są niezmienne.

%---------------------------------------------------------------------------
\section{Podstawy}
\label{sec:scala_basics}
Ta sekcja służy przybliżeniu czyletnikowi języka \textit{Scala} na poziomie wystatczającym aby swobodnie czytać przykłady
kodu umieszczone w tej pracy. W niektórych przykładach pomijane są przypadki skrajne lub nietypowe, celem szybkiego oraz 
jasnego przedstawienia minimum wiedzy na temat języka aby móc swobodnie go ,,czytać''.

\textit{Scala} jest językiem statycznie typowanym posiadającym lokalne ,,Type Inferrence''. Pozwala to kompilatorowi 
\textit{scalac} na ,,odnajdywanie'' typów wszystkich zmiennych oraz typów zwracanych przez metody podczas kompilacji,
bez potrzeby definiowania ich wprost. System ten

Użycie nawiasów \verb|()|, średnika \verb|;| oraz kropki \verb|.| jest analogiczne jak w przypadku Javy, 
jednak w wielu przypadkach opcjonalne gdyż kompilator jest w stanie wydedukować gdzie powinny się znaleźć.
\begin{lstlisting}
 val value = Option(42);
 val other = value.orElse(0);

 // moze zostac zastąpione
 val value = Option(42)
 val other = value orElse 0
\end{lstlisting}

Jednym z ciekawych przykładów stosowania notacji bez nawiasów i kropek jest \textit{ScalaTest}\footnote{ScalaTest - framework do testowania - http://www.scalatest.org}
(przy którego pomocy pisano testy w tym projekcie). Przykładowa \textit{asercja} napisana w \textit{DSL}u definiowanym przez tą bibliotekę wygląda następująco:
\begin{lstlisting}
 messages should (contain key ("Has") and not contain value ("NoSuchMsg"))
\end{lstlisting}

Dostępne jest wiele sposobów definiowania metod / pól w klasie,
w efekcie (na poziomie bytecode), wszystkie przekładane są na wywołania metod. Dostępne są słowa kluczowe:
\begin{itemize}
 \item \textbf{def}, \textit{def}iniujący zwyczajną metodę instancyjną. Warto nadmienić że Javowa koncepcja pojęcia \textit{static} nie jest dostępna z poziomu Scala.
 \item \textbf{val}, deklarujący ,,stałą'' - to jest metodę która raz zawołana, zwróci wartość oraz pole to będzie konsekwentnie zwracać tą samą wartość. 
              Dodatkowym efektem jest traktowanie zmiennych tego typu analogicznie do Javowych zmiennych z modyfikatorem \textbf{final}.
 \item \textbf{var}, deklaruje zwyczajną ,,zmienną'', do jakiej przyzwyczajeni jesteśmy z Java.
 \item modyfikator \textbf{lazy}, wpływający na moment inicjalizacji zmiennej - metody zadeklarowane z modyfikatorem \textbf{lazy} 
       zostaną dopiero zainicjalizowane podczas pierwszego odwołania się do tego pola z innego miejsca w kodzie. 
       W przypadku pary \textbf{lazy val}, metoda ta zostanie zawołana jedynie jednokrotnie, a zwrócona po raz pierwszy wartość
       zostanie zapisana w cache oraz będzie konsekwentnie zwracana podczas ponownych wywołań tej metody.
\end{itemize}

Modyfikator lazy pozwala na eleganckie budowy konstrukcji stosowanych w Parser Combinators, omówionych poniżej.


%---------------------------------------------------------------------------
\section{Traits - wmieszanie zachowania do klasy}
\label{sec:traits}

Słowo kluczowe \textbf{trait} rozpoczyna definicję typu zwanego traitem.
Implementacja nie różni się (na potrzeby tego szybkiego omówienia) od implementowania klasy,
jednak różnica jest podczas ,,dziedziczenia'' przy wykorzystaniu traitów. Nie mówimy bowiem o ,,dziedziczeniu'' 
w przypadku \textit{trait}ów, a o ,,wmieszaniu'' (ang. \textit{mixin} - wmieszanie) zachowania do klasy konkretnej.

Poniżej został przedstawiony najprostrzy trait zawierający jakieś zachowanie, oraz jeden ze sposobów jego wmieszania do klasy konkretnej.
Warto zauważyć że w przypadku wmieszania \textit{trait}a \verb|A| do klasy \verb|Test|, wprowadzamy między nimi relację ,,Test \textbf{IS-A} A'',
analogicznie jak w przypadku dziedziczenia.

\begin{lstlisting}
trait A { 
  def test = "A" // definicja metody zwracającej "A"
}

class Test extends A { } // wmieszanie A

new Test().test // skompiluje i wykona sie poprawnie
\end{lstlisting}

Co ciekawe, nie zauważamy różnicy w przypadku składni odnoszącej się do dziedziczenia dwóch klas konkretnych, oraz wmieszania traita.
Składnia ulega zmianie w przypadku korzystania z więcej niż jeden trait lub domieszania traita do klasy która już dziedziczy po innej klasie,
wówczas zamiast słowa kluczowego \textbf{extends} należy stosować \textbf{with} (nie dozwolone jest wielokrotne zapisanie \textbf{extends},
jednak wielokrotne \textbf{with} są często spotykane). Przykład wmieszania większej ilości traitów zostanie przedstawiony poniżej.

Jest to namiastka dziedziczenia wielobazowego jednak Scala dzięki swojemu bardzo rygorystycznemu kompilatorowi jest w stanie 
uniknąć sytuacji gdzie dziedziczenie wielobazowe byłoby niebezpieczne (klasyczne przykłady 
problematycznych sytuacji w przypadku dziedziczenia wielobazowago można przeczytać w ,,Symfonii C++'', autorstwa pana Grębosza \cite{symfonia}).

Kompilator \textit{scalac} przy napotkaniu konflintów nazw mogących doprowadzić do niejasności ,,którą metodę należy zawołać'', nie skompiluje takiego kodu
oraz poprosi o rozwiązanie konfliktu w sposób explicite. Jako przykład rozważmy dwa \textit{trait}y udostępniające metodę \verb|def test: String|:

\begin{lstlisting}
trait A { def test = "A" }
trait B { def test = "B" }

class Example extends A with B {
  // blad kompilacji!
}
\end{lstlisting}

Przy napotkaniu problemu tego typu kompilator zgłosi:
\begin{verbatim}
error: overriding method test in trait A of type => java.lang.String;
                  method test in trait B of type => java.lang.String needs `override' modifier
class Example extends A with B {
\end{verbatim}

Dzieje się tak ponieważ \textbf{scalac} próbuje odnaleźć która metoda powinna mieć większą wagę, a tymsamym powinna zostać wywołana.
Ponieważ nie jesteśmy w stanie dodać modyfikatora \textbf{override} do żadnego z \textit{trait}ów (ponieważ nie nadpisują one tej metody, a jedynie deklarują),
jedynym możliwym miejscem na rozwiązanie tego konfliktu jest uzupełnienie \verb|Example| o następujący fragment kodu, rutujący poprawnie nasze wywołanie metody:

\begin{lstlisting}
class Example extends A with B {
 // selektywne odwolanie sie do metody konkretnego supertypu
 override def test = super[B].test
}

new Example().test // poprawne
\end{lstlisting}

%---------------------------------------------------------------------------
\section{Case Class oraz Pattern Matching}
\label{sec:caseclass}

Klasy deklarowane przy pomocy \verb|case class| niosą ze sobą pewne ułatwienia, które generuje za nas kompilator.
Case klasy są wykorzystywane w bardzo wielu miejscach w ProtoDoc, ze względu na ich znaczną zwięzłość zapisu oraz wygodną współpracę z konstrukcją
\verb|match|, która zostanie za moment wyjaśniona.

Case klasę \small{(pozwolę sobie przyjąć taki, mieszający dwa języki, sposób nazywania jej, z racji problematycznego sesownego przetłumaczenia słowa case 
(ang. przypadek) w tym zwrocie)} definiujemy przy pomocy konstrukcji przedstawionej na Listingu \ref{lst:case_class}.

\begin{lstlisting}[caption={Deklaracja case klasy},label={lst:case_class}]
case class SampleProtoField(name: String, value: Long)
\end{lstlisting}

oraz odpowiada w przybliżeniu implementacji przedstawionej poniżej, na Listingu \ref{lst:case_class_by_hand}:

\begin{lstlisting}[caption={Ręczna implementacja case classy},label={lst:case_class_by_hand}]
class SampleProtoField {
 private def this(__name: String) = {
  this()
  _name = __name
 }

 private val _name = null

 def name =  _name

 def equals = /**/
 def hashCode = /**/
 def toString = /**/
}

object SampleProtoField {
 def apply(name: String) = new SampleProtoField(name)
 def unapply(field: SampleProtoField) = field.name
}
\end{lstlisting}

Dzieje się tutaj bardzo dużo ciekawych rzeczy sięgających głęboko po możliwości Scala, jednak w efekcie umożliwia nam:
\begin{itemize}
 \item domyślne utworzenie niezmiennych pól dla każdego argumentu w konstruktorze case klasy (val)
 \item automatyczne wygenerowanie getterów dla tych pól
 \item przyjemną dla oka implementację \verb|toString()|
 \item implementacje \verb|equals()| and \verb|hashCode()| (programiści Java znają ból generowania tych metod ręcznie)
 \item wygenerowanie ,,companion object'' pozwalającego na:
 \subitem tworzenie konstrukcji typu \verb|SampleProtoField("")| zamiast \verb|new SampleProtoField("")| (poprzez implementację apply)
 \subitem korzystanie z tej klasy w konstrukcji match (poprzez implementację unapply)
\end{itemize}

Najciekawsze dla nas są automatyczne implementacja apply / unapply, które w efekcie pozwalają na następujące linie:

\begin{lstlisting}
val it: SampleProtoField = SampleProtoField("name") // apply

val SampleProtoField(name) = "some name"
\end{lstlisting}

A jeżeli sięgnąć po konstrukcję match, pozwala ona na konstrukcje ,,wyjmujące'' wartości pól z skomplikowanych case class:

\begin{lstlisting}
myCaseClass match {
  case SampleProtoField(name) => println(name)
  case ComplicatedProtoField(name, type, _, _) => println(name +" & "+ type)
}
\end{lstlisting}

Konstrukcja \verb|match|, wywodzi się z programowania funkcyjnego. Mowa tutaj o ,,pattern matching''
znanym z chociażby \textbf{Erlanga}. Dzięki tej konstrukcji da się ominąć wiele linii kodu w stylu \verb|if(x) {x = x.getX(); method(x);}|.
Technika ta została zastosowana w bardzo wielu miejscach aplikacji, włącznie z parserem oraz weryfikatorem.


%---------------------------------------------------------------------------
\section{Implicit Conversions - konwersje ,,domniemane''}
\label{sec:implicit}
Scala pomimo że jest językiem silnie statycznie typowanym pozwala na pewne zabiegi aby ułatwić pracę w tak rygorystycznym type systemie.
Jednym z tych rozwiązań są tak zwane ,,Implicit Conversions'', będące typem metod, które kompilator może próbować zastosować podczas gdy 
potrzebna jest automatyczna konwersja z typu A na B. Najłatwiej będzie omówić to na prostym przykładzie (Listing \ref{lst:simple_implicit_example},
także spójrzmy na poniższe przypisanie liczby typu \verb|scala.Int| do zmiennej typu \verb|java.lang.String|:

\begin{lstlisting}[caption={Przykład wystąpienia implicit conversion}, label={lst:simple_implicit_example}]
val num: Int = 42
val string: String = num // compile time error!
\end{lstlisting}

Przykład przedstawiony na Listingu \ref{lst:simple_implicit_example} \textit{nie skompiluje się}, a kompilator odpowiedziałby następującym komunikatem:

\begin{verbatim}
 <console>:8: error: type mismatch;
 found   : Int
 required: String
       val string: String = num
\end{verbatim}

Dopisanie implicit konwersji w zasięgu widoczności tego przypisania, pozwoli natomiast kompilatorowi zauważyć iż dostępna jest metoda potrafiąca przeprowadzić
konwersję z typu \verb|Int | na \verb|String|, oraz ją zastosować. Implementację takiej konwersji przedstawiono na Listingu \ref{lst:an_implicit}.

\begin{lstlisting}[caption={Implementacja oraz zastosowanie konwersji domniemanej --- \textit{Implicit Conversion}}, label={lst:an_implicit}]
implicit def num2str(num: Int) = num.toString
// == implicit def num2str(num: Int): String = num.toString

val num: Int = 42
val string: String = num // ok!
\end{lstlisting}

To co się rzeczywiście dzieje podczas kompilacji, to zwyczajne wstawienie metody num2str, w linii z przypisaniem liczby do zmiennej typu String,
w następujący sposób: \verb|num2str(num)|. Istnieje więcej szczegółowych zasad dotyczących konwersji domniemanych, na przykład która konwersja powinna zostać 
zastosowana w przypadku większej ilości metod pozwalających na poprawne wykonanie przypisania, jednak nie będziemy w nie wnikać, ponieważ na potrzeby zrozumienia
zastosowanych DSL\footnote{DSL - Domain Specific Language} sama informacja o istnieniu tych konwersji powinna być wystarczająca.




%---------------------------------------------------------------------------
\section{Scala Parser Combinators}
\label{sec:scala_parser_combinators}
W tym rozdziale zostaną pokrótce przedstawione Scala Parser Combinators, oraz ich składnia.
Zostaną poruszone również tematy dotyczące wydajności kombinatorów oraz jak ustrzec się przez zanadto nawracającym się parserem.

Podstatową informacją którą należy sobie przyswoić podczas pracy z kombinatorami parserów, jest idea operowania na funkcjach wyższego rzędu - 
bo dokładnie tym są kombinatory. Najpierw jednak zdefiniujmy najprostszy możliwy parser (przedstawiony na Listingu \ref{lst:super_simple_parser}):

\begin{lstlisting}[caption={Najprostszy możliwy parser}, label={lst:super_simple_parser}]
object SimplestParser extends JavaTokenParsers {
  def a: Parser[String] = "a"
}
\end{lstlisting}

Przedstawiona na listingu imponująca implementacja parsera jednego znaku, de facto dokonywana jest przez implicit konwersję (dostarczoną przez \verb|JavaTokenParsers|), 
z typu \verb|String| na typ \verb|Parser[String]|. Zgodnie z oczekiwaniami ,,parsuje'' on jedynie jeden znak - oraz zwraca samego siebie (wspomniane ,,a'').
W przypadku napisu dłuższego niż samo ,,a'', prymitywny parser możemy zdefiniować jako funkcję przyjmującą napis wejściowy a zwracającą krotkę 
przeprasowanego elementu, oraz pozostałej części napisu. W szczególności Parser zatem można zdefiniować jako:

\begin{center}
\begin{tabular}{|p{\textwidth}|}
\hline
\textbf{Definicja 1} \\
\textit{Parser} zdefiniowany jest jako funkcja przyjmująca napis wejściowy, a zwracająca krotkę sparsowanego elementu oraz pozostałego napisu wejściowego. \\ 
\cite{monadparsing}
\\ \hline
\end{tabular}
\end{center}

Mając tak zdefiniowany parser, możemy przedstawić definicję kombinatora parserów:

\begin{center}
\begin{tabular}{|p{\textwidth}|}
\hline
\textbf{Definicja 2} \\
\textit{Kombinator parserów}, jest \textit{funkcją wyższego rzędu},
która przyjmuje jako argumenty funkcje parsujące, oraz zwraca nową funkcję utworzoną poprzez ich kombinację -- zależną od jego implementacji. \\

\cite{monadparsing}
\\ \hline
\end{tabular}
\end{center}

W przypadku Scali, kombinatory implementowane są poprzez metody wypisane w Tabeli \ref{combinators_table}, zdefiniowane w typie \verb|Parser|.

\begin{table}
\begin{center}
\begin{tabular}{|l|l|p{7cm}|}
\hline
Metoda & Nazwa & Przykład \\ \hline
\verb|~| & sekwencja & \verb|"a" ~ "b"| parsuje ciąg ,,ab'', oraz zwraca ,,ab''\\
\verb|~>| & sekwencja, odrzucając lewą stronę & \verb|,,a'' ~> ,,b''| parsuje ciąg ,,ab'', oraz zwraca ,,b''\\
\verb|<~| & sekwencja, odrzucając prawą stronę & \verb|,,a'' <~ ,,b''| parsuje ciąg ,,ab'', oraz zwraca ,,a''\\
\verb\|\ & alternatywa & ,,a'' | ,,b'' parsuje ciąg ,,a'' lub ,,b'', oraz zwraca ,,a'' lub ,,b''\\
\verb\^^\ & transformacja & \verb|,,a'' ^^ () => Any| parsuje ciąg ,,a'', oraz zwraca do, co zwróci funkcja przekazana jako argument tego kombinatora. Zostanie opisany szczegółowo poniżej. \\
\verb|opt()| & sekwencja & opt(,,b'') parsuje ciąg ,,'' lub ,,b'', oraz zwraca Option[String], mogące zawierać ,,b'' lub nic\\
\verb|rep()| & powtórzenie & rep(,,b'') parsuje ciąg składający się z powtórzeń ciągu ,,b'', oraz zwraca List(,,b'', ,,b'', ,,b'', ...)]\\
\verb|repsep()| & powtórzenie, z separatorami & repsep(,,b'', ,,,'') parsuje ciąg składający się z powtórzeń ciągu ,,b'' oddzielonych znakiem ,,,'', oraz zwraca List(,,b'', ,,b'', ,,b'', ...)]\\
\verb|rep()| & powtórzenie & rep(,,b'') parsuje ciąg składający się z powtórzeń ciągu ,,b'', oraz zwraca List(,,b'', ,,b'', ,,b'', ...)]\\ \hline
\end{tabular}
\end{center}
\caption{Dostępne podstawowe kombinatory parserów}
\label{combinators_table}
\end{table}

Warto tutaj przytoczyć implementację dwóch kombinatorów, ponieważ pokazują jak olbrzymią elastyczność w definiowaniu ich uzyskać
poprzez wykorzystywanie bardzo prostych kombinatorów (wewnętrznie zdefiniowanych \verb\|||\ oraz \verb|&&&|, z których jako programista korzystający
z Scala Parser Combinators raczej korzystać nie będziemy). Warte zauważenia na Listingu \ref{optrep} jest na przykład wywołanie \verb|rep(p)|,
otóż dzięki elastyczności Scali, możliwe jest zaimplementowanie Parserów w ten sposób że do wywołania rep(p) nie koniecznie musi dojść, pomimo że 
mamy do czynienia z zwyczajnymi wywołaniami metod. Umożliwia to tak zwany ,,pass-by-name parameter'', jednak znacznie wykracza to poza zakres konieczny 
do zrozumienia działania Parser Combinators, także pozostaniemy przy stwierdzeniu iż mamy tutaj do czynienia z leniwą ewaluacją tej metody.

\begin{lstlisting}[caption={Implementacja parserów opt i rep}, label={optrep}]
def opt(p: Parser): Parser = p ||| empty;   // p ? p : empty
def rep(p: Parser): Parser = opt(rep1(p));  // p* = [p+]
def rep1(p: Parser): Parser = p &&& rep(p); // p+ = p ~ p*
\end{lstlisting}


Przy odpowiednim zastosowaniu tych prostych zasad, można bardzo łatwo tworzyć parsery nawet skomplikowanych gramatyk.
Niestety z natury działania kombinatorów, możliwe jest wygenerowanie parsera który będzie wykonywał wielokrotne oraz głębokie nawroty.
Celem zmniejszenia miejsc narażenia się na zbyt dużą klasę wygenerowanego parsera \textit{LL(k)}, możemy zastosować kombinator \verb|~!|
który zgłosi błąd jeżeli dana kombinacja nie jest możliwa do rozwiązania w sposób generujący Parser klasy \textit{LL(1)} w tym miejscu.
Powinno się stosować ten operator jeżeli tylko możliwe, aby zagwarantować sobie wczesne ostrzeżenie o nie optymalnej formie gramatyki.
Wówczas można przeprowadzić faktoryzację lewostronną, celem wyciągnięcia wspólnego symbolu dla dwóch lub więcej produkcji ,,przed nawias'', zmniejszając klasę parsera.

Ostatnim elementem koniecznym do wprowadzenia zanim przejdziemy do przykładów pełnych implementacji prostych parserów, jest kombinator \verb|^^|,
oznaczający ,,\textit{transformację}''. Kombinator ten służyć nam będzie do produkowania typów które chcemy produkować z danego sparsowanego elementu,
nie natomiast prymitówów takich jak listy i typy podstawowe, które zostałyby zwrócone przez parser bez zastosowania tego kombinatora.
Prostym przykładem jest parser przedstawiony na Listingu \ref{so_simple}, który transformuje napis ,,true'' na wartość typu \verb|Boolean|.

\begin{lstlisting}[caption={Zastosowanie kombinatora transformującego}, label={so_simple}]
def TRUE: Parser[Boolean] = "true" ^^ { b => true }

//                                      without ^^, 
// def TRUE: Parser[String] = "true" // value is String!

\end{lstlisting}

\subsection{Przykłady parserów}
Przedstawione zostaną dwa proste przykłady parserów, mieszczące się na 1 stronie A4, oraz realizujące przydatne operacje, 
takie jak parsowanie formatu JSON, oraz obliczenie prostego wyrażenia matematycznego. Przykłady te mogą posłyżuć jako pomoc w unaocznieniu elegancji
oraz potęgi Parser Combinators.

Jako pierwszy przykład parsera chciałbym w tym miejscu zacytować implementację parsera JSON \footnote{JSON - Java Script Object Notation} umieszczonego w książce
\textit{Programming in Scala} autorstwa Martina Oderskiego \cite{odersky_scala}:

\begin{lstlisting}
import scala.util.parsing.combinator._

class JSON extends JavaTokenParsers {   

  def obj: Parser[Map[String, Any]] = 
    "{"~> repsep(member, ",") <~"}" ^^ (Map() ++ _)
    
  def arr: Parser[List[Any]] =
    "["~> repsep(value, ",") <~"]" 

  def member: Parser[(String, Any)] = 
    stringLiteral~":"~value ^^ 
      { case name~":"~value => (name, value) }

  def value: Parser[Any] = (
    obj
  | arr 
  | stringLiteral
  | floatingPointNumber ^^ (_.toDouble) 
  | "null"  ^^ (x => null) 
  | "true"  ^^ (x => true) 
  | "false" ^^ (x => false)
  )
}
\end{lstlisting}

Kod nie dość że jest elegancki, to okazuje się, gramatyka wykorzystana w tym przykładzie (a zatem i wygenerowany parser) jest klasy \textit{LL(1)}.
Oznacza to, że kosztowny mechanizm nawrotów nigdy nie zostanie wykorzystany podczas parsowania JSONa przy pomocy powyższego parsera. Jest to ciekawy argument
przeciwko zarzutowi stawianemu kombinatorom parserów, iż generują bardzo powolne (nawracające się) parsery -- jest to oczywiście nie prawda, ponieważ to 
od zastosowanej gramatyki zależy jakiej klasy otrzymamy w efekcie parser.

~\\\*

Drugim przykładem parsera jest zaimplementowany przeze mnie celem pokazania kombinatorów parserów prosty parser,
który w efekcie działania oblicza b. proste wyrażenia matematyczne. Parser ten można raczej traktować jako ciekawostkę niż pełną implementację Idei, 
jednak obrazuje on bardzo dobrze jak wykorzystywane są operacje \verb|^^| celem wpasowania parsowanego drzewa do własnego modelu -- analogicznie
jak miało to miejsce podczas implementowania parsera ProtoDoc.

\begin{lstlisting}
object MathParser extends RegexParsers with ImplicitConversions {
  def number: Parser[Long] = "[0-9]+".r ^^ { _.toLong }
  def num: Parser[Long] = lp ~> number <~ rp
  
  def add: Parser[AOperation] = ("+" | "-" ) ^^ { AOperation(_) }
  def mult: Parser[MOperation] = ("*" | "/" ) ^^ { MOperation(_) }

  def op: Parser[Long] = lp ~> num ~! (add | mult) ~! num <~ rp ^^ {
    case num1 ~ op ~ num2 => op.perform(num1, num2)
  }

  def lp: Parser[Option[String]] = opt("(")
  def rp: Parser[Option[String]] = opt(")")
  def stop: Parser[Option[String]] = opt(";")
  
  def parse(string: String): Any = parseAll(op, string) match {
    case Success(res, _) => res
    case e => throw new RuntimeException(e.toString)
  }
}

object MathParserTest extends App {
  override def main(args: Array[String]) {
    import MathParser._
    println(parse("10+15")) // 25
    println(parse("10*2")) // 20
  }
}
\end{lstlisting}


Kod ponownie jest zwięzły, czytelny oraz scentralizowany w jednym miejscu. W ramach przypomnienia jak podobny parser wyglądałby w GNU Bison, 
można spojrzeć tutaj: \href{http://www.usna.edu/Users/cs/lmcdowel/courses/si413/F10/labs/L04/calc1/ex1.html}{http://www.usna.edu/Users/cs/lmcdowel/courses/si413/F10/labs/L04/calc1/ex1.html}
gdzie umieszczony został przykład podobnego parsera. Wadami rozwiązania problemy przy pomocy Bison oraz Flex jest korzystanie z wielu narzędzi,
wielu plików, oraz wielokrokowy proces budowania projektu -- co w kontraście do 1 pliku, 1 języka, oraz braku dodatkowych kroków, faktycznie daje Scali pewną przewagę.